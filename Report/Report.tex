\documentclass[11pt]{article}

\usepackage{fontspec}
\setmainfont{Arial}

    \usepackage[breakable]{tcolorbox}
    \usepackage{parskip} % Stop auto-indenting (to mimic markdown behaviour)
\usepackage{underscore}

\usepackage{listings}
\usepackage{fontspec}
\setmonofont{Arial}
\lstset{
  basicstyle=\ttfamily,  % Set the font to monospace
  keywordstyle=\bfseries,  % Customize keyword style if desired
  % Add more options as needed for your code style
}
  

    % Basic figure setup, for now with no caption control since it's done
    % automatically by Pandoc (which extracts ![](path) syntax from Markdown).
    \usepackage{graphicx}
    % Maintain compatibility with old templates. Remove in nbconvert 6.0
    \let\Oldincludegraphics\includegraphics
    % Ensure that by default, figures have no caption (until we provide a
    % proper Figure object with a Caption API and a way to capture that
    % in the conversion process - todo).
    \usepackage{caption}
    \DeclareCaptionFormat{nocaption}{}
    \captionsetup{format=nocaption,aboveskip=0pt,belowskip=0pt}

    \usepackage{float}
    \floatplacement{figure}{H} % forces figures to be placed at the correct location
    \usepackage{xcolor} % Allow colors to be defined
    \usepackage{enumerate} % Needed for markdown enumerations to work
    \usepackage{geometry} % Used to adjust the document margins
    \usepackage{amsmath} % Equations
    \usepackage{amssymb} % Equations
    \usepackage{textcomp} % defines textquotesingle
    % Hack from http://tex.stackexchange.com/a/47451/13684:
    \AtBeginDocument{%
        \def\PYZsq{\textquotesingle}% Upright quotes in Pygmentized code
    }
    \usepackage{upquote} % Upright quotes for verbatim code
    \usepackage{eurosym} % defines \euro

    \usepackage{iftex}
    \ifPDFTeX
        \usepackage[T1]{fontenc}
        \IfFileExists{alphabeta.sty}{
              \usepackage{alphabeta}
          }{
              \usepackage[mathletters]{ucs}
              \usepackage[utf8x]{inputenc}
          }
    \else
        \usepackage{fontspec}
        \usepackage{unicode-math}
    \fi

    \usepackage{fancyvrb} % verbatim replacement that allows latex
    \usepackage{grffile} % extends the file name processing of package graphics
                         % to support a larger range
    \makeatletter % fix for old versions of grffile with XeLaTeX
    \@ifpackagelater{grffile}{2019/11/01}
    {
      % Do nothing on new versions
    }
    {
      \def\Gread@@xetex#1{%
        \IfFileExists{"\Gin@base".bb}%
        {\Gread@eps{\Gin@base.bb}}%
        {\Gread@@xetex@aux#1}%
      }
    }
    \makeatother
    \usepackage[Export]{adjustbox} % Used to constrain images to a maximum size
    \adjustboxset{max size={0.9\linewidth}{0.9\paperheight}}

    % The hyperref package gives us a pdf with properly built
    % internal navigation ('pdf bookmarks' for the table of contents,
    % internal cross-reference links, web links for URLs, etc.)
    \usepackage{hyperref}
    % The default LaTeX title has an obnoxious amount of whitespace. By default,
    % titling removes some of it. It also provides customization options.
    \usepackage{titling}
    \usepackage{longtable} % longtable support required by pandoc >1.10
    \usepackage{booktabs}  % table support for pandoc > 1.12.2
    \usepackage{array}     % table support for pandoc >= 2.11.3
    \usepackage{calc}      % table minipage width calculation for pandoc >= 2.11.1
    \usepackage[inline]{enumitem} % IRkernel/repr support (it uses the enumerate* environment)
    \usepackage[normalem]{ulem} % ulem is needed to support strikethroughs (\sout)
                                % normalem makes italics be italics, not underlines
    \usepackage{mathrsfs}
    

    
    % Colors for the hyperref package
    \definecolor{urlcolor}{rgb}{0,.145,.698}
    \definecolor{linkcolor}{rgb}{.71,0.21,0.01}
    \definecolor{citecolor}{rgb}{.12,.54,.11}

    % ANSI colors
    \definecolor{ansi-black}{HTML}{3E424D}
    \definecolor{ansi-black-intense}{HTML}{282C36}
    \definecolor{ansi-red}{HTML}{E75C58}
    \definecolor{ansi-red-intense}{HTML}{B22B31}
    \definecolor{ansi-green}{HTML}{00A250}
    \definecolor{ansi-green-intense}{HTML}{007427}
    \definecolor{ansi-yellow}{HTML}{DDB62B}
    \definecolor{ansi-yellow-intense}{HTML}{B27D12}
    \definecolor{ansi-blue}{HTML}{208FFB}
    \definecolor{ansi-blue-intense}{HTML}{0065CA}
    \definecolor{ansi-magenta}{HTML}{D160C4}
    \definecolor{ansi-magenta-intense}{HTML}{A03196}
    \definecolor{ansi-cyan}{HTML}{60C6C8}
    \definecolor{ansi-cyan-intense}{HTML}{258F8F}
    \definecolor{ansi-white}{HTML}{C5C1B4}
    \definecolor{ansi-white-intense}{HTML}{A1A6B2}
    \definecolor{ansi-default-inverse-fg}{HTML}{FFFFFF}
    \definecolor{ansi-default-inverse-bg}{HTML}{000000}

    % common color for the border for error outputs.
    \definecolor{outerrorbackground}{HTML}{FFDFDF}

    % commands and environments needed by pandoc snippets
    % extracted from the output of `pandoc -s`
    \providecommand{\tightlist}{%
      \setlength{\itemsep}{0pt}\setlength{\parskip}{0pt}}
    \DefineVerbatimEnvironment{Highlighting}{Verbatim}{commandchars=\\\{\}}
    % Add ',fontsize=\small' for more characters per line
    \newenvironment{Shaded}{}{}
    \newcommand{\KeywordTok}[1]{\textcolor[rgb]{0.00,0.44,0.13}{\textbf{{#1}}}}
    \newcommand{\DataTypeTok}[1]{\textcolor[rgb]{0.56,0.13,0.00}{{#1}}}
    \newcommand{\DecValTok}[1]{\textcolor[rgb]{0.25,0.63,0.44}{{#1}}}
    \newcommand{\BaseNTok}[1]{\textcolor[rgb]{0.25,0.63,0.44}{{#1}}}
    \newcommand{\FloatTok}[1]{\textcolor[rgb]{0.25,0.63,0.44}{{#1}}}
    \newcommand{\CharTok}[1]{\textcolor[rgb]{0.25,0.44,0.63}{{#1}}}
    \newcommand{\StringTok}[1]{\textcolor[rgb]{0.25,0.44,0.63}{{#1}}}
    \newcommand{\CommentTok}[1]{\textcolor[rgb]{0.38,0.63,0.69}{\textit{{#1}}}}
    \newcommand{\OtherTok}[1]{\textcolor[rgb]{0.00,0.44,0.13}{{#1}}}
    \newcommand{\AlertTok}[1]{\textcolor[rgb]{1.00,0.00,0.00}{\textbf{{#1}}}}
    \newcommand{\FunctionTok}[1]{\textcolor[rgb]{0.02,0.16,0.49}{{#1}}}
    \newcommand{\RegionMarkerTok}[1]{{#1}}
    \newcommand{\ErrorTok}[1]{\textcolor[rgb]{1.00,0.00,0.00}{\textbf{{#1}}}}
    \newcommand{\NormalTok}[1]{{#1}}

    % Additional commands for more recent versions of Pandoc
    \newcommand{\ConstantTok}[1]{\textcolor[rgb]{0.53,0.00,0.00}{{#1}}}
    \newcommand{\SpecialCharTok}[1]{\textcolor[rgb]{0.25,0.44,0.63}{{#1}}}
    \newcommand{\VerbatimStringTok}[1]{\textcolor[rgb]{0.25,0.44,0.63}{{#1}}}
    \newcommand{\SpecialStringTok}[1]{\textcolor[rgb]{0.73,0.40,0.53}{{#1}}}
    \newcommand{\ImportTok}[1]{{#1}}
    \newcommand{\DocumentationTok}[1]{\textcolor[rgb]{0.73,0.13,0.13}{\textit{{#1}}}}
    \newcommand{\AnnotationTok}[1]{\textcolor[rgb]{0.38,0.63,0.69}{\textbf{\textit{{#1}}}}}
    \newcommand{\CommentVarTok}[1]{\textcolor[rgb]{0.38,0.63,0.69}{\textbf{\textit{{#1}}}}}
    \newcommand{\VariableTok}[1]{\textcolor[rgb]{0.10,0.09,0.49}{{#1}}}
    \newcommand{\ControlFlowTok}[1]{\textcolor[rgb]{0.00,0.44,0.13}{\textbf{{#1}}}}
    \newcommand{\OperatorTok}[1]{\textcolor[rgb]{0.40,0.40,0.40}{{#1}}}
    \newcommand{\BuiltInTok}[1]{{#1}}
    \newcommand{\ExtensionTok}[1]{{#1}}
    \newcommand{\PreprocessorTok}[1]{\textcolor[rgb]{0.74,0.48,0.00}{{#1}}}
    \newcommand{\AttributeTok}[1]{\textcolor[rgb]{0.49,0.56,0.16}{{#1}}}
    \newcommand{\InformationTok}[1]{\textcolor[rgb]{0.38,0.63,0.69}{\textbf{\textit{{#1}}}}}
    \newcommand{\WarningTok}[1]{\textcolor[rgb]{0.38,0.63,0.69}{\textbf{\textit{{#1}}}}}


    % Define a nice break command that doesn't care if a line doesn't already
    % exist.
    \def\br{\hspace*{\fill} \\* }
    % Math Jax compatibility definitions
    \def\gt{>}
    \def\lt{<}
    \let\Oldtex\TeX
    \let\Oldlatex\LaTeX
    \renewcommand{\TeX}{\textrm{\Oldtex}}
    \renewcommand{\LaTeX}{\textrm{\Oldlatex}}
    % Document parameters
    % Document title
    \title{Datenanalyse- und visualisierung von Tankstellen und dessen Kraftstoffpreisdaten über den Zeitraum 2014 - April 2023 in Deutschland}
\author{Ahmed Dhahir, Laelia Chi, Ole Schildt und Philipp Meyer}
\date{\today}    
    
    
    
    
% Pygments definitions
\makeatletter
\def\PY@reset{\let\PY@it=\relax \let\PY@bf=\relax%
    \let\PY@ul=\relax \let\PY@tc=\relax%
    \let\PY@bc=\relax \let\PY@ff=\relax}
\def\PY@tok#1{\csname PY@tok@#1\endcsname}
\def\PY@toks#1+{\ifx\relax#1\empty\else%
    \PY@tok{#1}\expandafter\PY@toks\fi}
\def\PY@do#1{\PY@bc{\PY@tc{\PY@ul{%
    \PY@it{\PY@bf{\PY@ff{#1}}}}}}}
\def\PY#1#2{\PY@reset\PY@toks#1+\relax+\PY@do{#2}}

\@namedef{PY@tok@w}{\def\PY@tc##1{\textcolor[rgb]{0.73,0.73,0.73}{##1}}}
\@namedef{PY@tok@c}{\let\PY@it=\textit\def\PY@tc##1{\textcolor[rgb]{0.24,0.48,0.48}{##1}}}
\@namedef{PY@tok@cp}{\def\PY@tc##1{\textcolor[rgb]{0.61,0.40,0.00}{##1}}}
\@namedef{PY@tok@k}{\let\PY@bf=\textbf\def\PY@tc##1{\textcolor[rgb]{0.00,0.50,0.00}{##1}}}
\@namedef{PY@tok@kp}{\def\PY@tc##1{\textcolor[rgb]{0.00,0.50,0.00}{##1}}}
\@namedef{PY@tok@kt}{\def\PY@tc##1{\textcolor[rgb]{0.69,0.00,0.25}{##1}}}
\@namedef{PY@tok@o}{\def\PY@tc##1{\textcolor[rgb]{0.40,0.40,0.40}{##1}}}
\@namedef{PY@tok@ow}{\let\PY@bf=\textbf\def\PY@tc##1{\textcolor[rgb]{0.67,0.13,1.00}{##1}}}
\@namedef{PY@tok@nb}{\def\PY@tc##1{\textcolor[rgb]{0.00,0.50,0.00}{##1}}}
\@namedef{PY@tok@nf}{\def\PY@tc##1{\textcolor[rgb]{0.00,0.00,1.00}{##1}}}
\@namedef{PY@tok@nc}{\let\PY@bf=\textbf\def\PY@tc##1{\textcolor[rgb]{0.00,0.00,1.00}{##1}}}
\@namedef{PY@tok@nn}{\let\PY@bf=\textbf\def\PY@tc##1{\textcolor[rgb]{0.00,0.00,1.00}{##1}}}
\@namedef{PY@tok@ne}{\let\PY@bf=\textbf\def\PY@tc##1{\textcolor[rgb]{0.80,0.25,0.22}{##1}}}
\@namedef{PY@tok@nv}{\def\PY@tc##1{\textcolor[rgb]{0.10,0.09,0.49}{##1}}}
\@namedef{PY@tok@no}{\def\PY@tc##1{\textcolor[rgb]{0.53,0.00,0.00}{##1}}}
\@namedef{PY@tok@nl}{\def\PY@tc##1{\textcolor[rgb]{0.46,0.46,0.00}{##1}}}
\@namedef{PY@tok@ni}{\let\PY@bf=\textbf\def\PY@tc##1{\textcolor[rgb]{0.44,0.44,0.44}{##1}}}
\@namedef{PY@tok@na}{\def\PY@tc##1{\textcolor[rgb]{0.41,0.47,0.13}{##1}}}
\@namedef{PY@tok@nt}{\let\PY@bf=\textbf\def\PY@tc##1{\textcolor[rgb]{0.00,0.50,0.00}{##1}}}
\@namedef{PY@tok@nd}{\def\PY@tc##1{\textcolor[rgb]{0.67,0.13,1.00}{##1}}}
\@namedef{PY@tok@s}{\def\PY@tc##1{\textcolor[rgb]{0.73,0.13,0.13}{##1}}}
\@namedef{PY@tok@sd}{\let\PY@it=\textit\def\PY@tc##1{\textcolor[rgb]{0.73,0.13,0.13}{##1}}}
\@namedef{PY@tok@si}{\let\PY@bf=\textbf\def\PY@tc##1{\textcolor[rgb]{0.64,0.35,0.47}{##1}}}
\@namedef{PY@tok@se}{\let\PY@bf=\textbf\def\PY@tc##1{\textcolor[rgb]{0.67,0.36,0.12}{##1}}}
\@namedef{PY@tok@sr}{\def\PY@tc##1{\textcolor[rgb]{0.64,0.35,0.47}{##1}}}
\@namedef{PY@tok@ss}{\def\PY@tc##1{\textcolor[rgb]{0.10,0.09,0.49}{##1}}}
\@namedef{PY@tok@sx}{\def\PY@tc##1{\textcolor[rgb]{0.00,0.50,0.00}{##1}}}
\@namedef{PY@tok@m}{\def\PY@tc##1{\textcolor[rgb]{0.40,0.40,0.40}{##1}}}
\@namedef{PY@tok@gh}{\let\PY@bf=\textbf\def\PY@tc##1{\textcolor[rgb]{0.00,0.00,0.50}{##1}}}
\@namedef{PY@tok@gu}{\let\PY@bf=\textbf\def\PY@tc##1{\textcolor[rgb]{0.50,0.00,0.50}{##1}}}
\@namedef{PY@tok@gd}{\def\PY@tc##1{\textcolor[rgb]{0.63,0.00,0.00}{##1}}}
\@namedef{PY@tok@gi}{\def\PY@tc##1{\textcolor[rgb]{0.00,0.52,0.00}{##1}}}
\@namedef{PY@tok@gr}{\def\PY@tc##1{\textcolor[rgb]{0.89,0.00,0.00}{##1}}}
\@namedef{PY@tok@ge}{\let\PY@it=\textit}
\@namedef{PY@tok@gs}{\let\PY@bf=\textbf}
\@namedef{PY@tok@gp}{\let\PY@bf=\textbf\def\PY@tc##1{\textcolor[rgb]{0.00,0.00,0.50}{##1}}}
\@namedef{PY@tok@go}{\def\PY@tc##1{\textcolor[rgb]{0.44,0.44,0.44}{##1}}}
\@namedef{PY@tok@gt}{\def\PY@tc##1{\textcolor[rgb]{0.00,0.27,0.87}{##1}}}
\@namedef{PY@tok@err}{\def\PY@bc##1{{\setlength{\fboxsep}{\string -\fboxrule}\fcolorbox[rgb]{1.00,0.00,0.00}{1,1,1}{\strut ##1}}}}
\@namedef{PY@tok@kc}{\let\PY@bf=\textbf\def\PY@tc##1{\textcolor[rgb]{0.00,0.50,0.00}{##1}}}
\@namedef{PY@tok@kd}{\let\PY@bf=\textbf\def\PY@tc##1{\textcolor[rgb]{0.00,0.50,0.00}{##1}}}
\@namedef{PY@tok@kn}{\let\PY@bf=\textbf\def\PY@tc##1{\textcolor[rgb]{0.00,0.50,0.00}{##1}}}
\@namedef{PY@tok@kr}{\let\PY@bf=\textbf\def\PY@tc##1{\textcolor[rgb]{0.00,0.50,0.00}{##1}}}
\@namedef{PY@tok@bp}{\def\PY@tc##1{\textcolor[rgb]{0.00,0.50,0.00}{##1}}}
\@namedef{PY@tok@fm}{\def\PY@tc##1{\textcolor[rgb]{0.00,0.00,1.00}{##1}}}
\@namedef{PY@tok@vc}{\def\PY@tc##1{\textcolor[rgb]{0.10,0.09,0.49}{##1}}}
\@namedef{PY@tok@vg}{\def\PY@tc##1{\textcolor[rgb]{0.10,0.09,0.49}{##1}}}
\@namedef{PY@tok@vi}{\def\PY@tc##1{\textcolor[rgb]{0.10,0.09,0.49}{##1}}}
\@namedef{PY@tok@vm}{\def\PY@tc##1{\textcolor[rgb]{0.10,0.09,0.49}{##1}}}
\@namedef{PY@tok@sa}{\def\PY@tc##1{\textcolor[rgb]{0.73,0.13,0.13}{##1}}}
\@namedef{PY@tok@sb}{\def\PY@tc##1{\textcolor[rgb]{0.73,0.13,0.13}{##1}}}
\@namedef{PY@tok@sc}{\def\PY@tc##1{\textcolor[rgb]{0.73,0.13,0.13}{##1}}}
\@namedef{PY@tok@dl}{\def\PY@tc##1{\textcolor[rgb]{0.73,0.13,0.13}{##1}}}
\@namedef{PY@tok@s2}{\def\PY@tc##1{\textcolor[rgb]{0.73,0.13,0.13}{##1}}}
\@namedef{PY@tok@sh}{\def\PY@tc##1{\textcolor[rgb]{0.73,0.13,0.13}{##1}}}
\@namedef{PY@tok@s1}{\def\PY@tc##1{\textcolor[rgb]{0.73,0.13,0.13}{##1}}}
\@namedef{PY@tok@mb}{\def\PY@tc##1{\textcolor[rgb]{0.40,0.40,0.40}{##1}}}
\@namedef{PY@tok@mf}{\def\PY@tc##1{\textcolor[rgb]{0.40,0.40,0.40}{##1}}}
\@namedef{PY@tok@mh}{\def\PY@tc##1{\textcolor[rgb]{0.40,0.40,0.40}{##1}}}
\@namedef{PY@tok@mi}{\def\PY@tc##1{\textcolor[rgb]{0.40,0.40,0.40}{##1}}}
\@namedef{PY@tok@il}{\def\PY@tc##1{\textcolor[rgb]{0.40,0.40,0.40}{##1}}}
\@namedef{PY@tok@mo}{\def\PY@tc##1{\textcolor[rgb]{0.40,0.40,0.40}{##1}}}
\@namedef{PY@tok@ch}{\let\PY@it=\textit\def\PY@tc##1{\textcolor[rgb]{0.24,0.48,0.48}{##1}}}
\@namedef{PY@tok@cm}{\let\PY@it=\textit\def\PY@tc##1{\textcolor[rgb]{0.24,0.48,0.48}{##1}}}
\@namedef{PY@tok@cpf}{\let\PY@it=\textit\def\PY@tc##1{\textcolor[rgb]{0.24,0.48,0.48}{##1}}}
\@namedef{PY@tok@c1}{\let\PY@it=\textit\def\PY@tc##1{\textcolor[rgb]{0.24,0.48,0.48}{##1}}}
\@namedef{PY@tok@cs}{\let\PY@it=\textit\def\PY@tc##1{\textcolor[rgb]{0.24,0.48,0.48}{##1}}}

\def\PYZbs{\char`\\}
\def\PYZus{\char`\_}
\def\PYZob{\char`\{}
\def\PYZcb{\char`\}}
\def\PYZca{\char`\^}
\def\PYZam{\char`\&}
\def\PYZlt{\char`\<}
\def\PYZgt{\char`\>}
\def\PYZsh{\char`\#}
\def\PYZpc{\char`\%}
\def\PYZdl{\char`\$}
\def\PYZhy{\char`\-}
\def\PYZsq{\char`\'}
\def\PYZdq{\char`\"}
\def\PYZti{\char`\~}
% for compatibility with earlier versions
\def\PYZat{@}
\def\PYZlb{[}
\def\PYZrb{]}
\makeatother


    % For linebreaks inside Verbatim environment from package fancyvrb.
    \makeatletter
        \newbox\Wrappedcontinuationbox
        \newbox\Wrappedvisiblespacebox
        \newcommand*\Wrappedvisiblespace {\textcolor{red}{\textvisiblespace}}
        \newcommand*\Wrappedcontinuationsymbol {\textcolor{red}{\llap{\tiny$\m@th\hookrightarrow$}}}
        \newcommand*\Wrappedcontinuationindent {3ex }
        \newcommand*\Wrappedafterbreak {\kern\Wrappedcontinuationindent\copy\Wrappedcontinuationbox}
        % Take advantage of the already applied Pygments mark-up to insert
        % potential linebreaks for TeX processing.
        %        {, <, #, %, $, ' and ": go to next line.
        %        _, }, ^, &, >, - and ~: stay at end of broken line.
        % Use of \textquotesingle for straight quote.
        \newcommand*\Wrappedbreaksatspecials {%
            \def\PYGZus{\discretionary{\char`\_}{\Wrappedafterbreak}{\char`\_}}%
            \def\PYGZob{\discretionary{}{\Wrappedafterbreak\char`\{}{\char`\{}}%
            \def\PYGZcb{\discretionary{\char`\}}{\Wrappedafterbreak}{\char`\}}}%
            \def\PYGZca{\discretionary{\char`\^}{\Wrappedafterbreak}{\char`\^}}%
            \def\PYGZam{\discretionary{\char`\&}{\Wrappedafterbreak}{\char`\&}}%
            \def\PYGZlt{\discretionary{}{\Wrappedafterbreak\char`\<}{\char`\<}}%
            \def\PYGZgt{\discretionary{\char`\>}{\Wrappedafterbreak}{\char`\>}}%
            \def\PYGZsh{\discretionary{}{\Wrappedafterbreak\char`\#}{\char`\#}}%
            \def\PYGZpc{\discretionary{}{\Wrappedafterbreak\char`\%}{\char`\%}}%
            \def\PYGZdl{\discretionary{}{\Wrappedafterbreak\char`\$}{\char`\$}}%
            \def\PYGZhy{\discretionary{\char`\-}{\Wrappedafterbreak}{\char`\-}}%
            \def\PYGZsq{\discretionary{}{\Wrappedafterbreak\textquotesingle}{\textquotesingle}}%
            \def\PYGZdq{\discretionary{}{\Wrappedafterbreak\char`\"}{\char`\"}}%
            \def\PYGZti{\discretionary{\char`\~}{\Wrappedafterbreak}{\char`\~}}%
        }
        % Some characters . , ; ? ! / are not pygmentized.
        % This macro makes them "active" and they will insert potential linebreaks
        \newcommand*\Wrappedbreaksatpunct {%
            \lccode`\~`\.\lowercase{\def~}{\discretionary{\hbox{\char`\.}}{\Wrappedafterbreak}{\hbox{\char`\.}}}%
            \lccode`\~`\,\lowercase{\def~}{\discretionary{\hbox{\char`\,}}{\Wrappedafterbreak}{\hbox{\char`\,}}}%
            \lccode`\~`\;\lowercase{\def~}{\discretionary{\hbox{\char`\;}}{\Wrappedafterbreak}{\hbox{\char`\;}}}%
            \lccode`\~`\:\lowercase{\def~}{\discretionary{\hbox{\char`\:}}{\Wrappedafterbreak}{\hbox{\char`\:}}}%
            \lccode`\~`\?\lowercase{\def~}{\discretionary{\hbox{\char`\?}}{\Wrappedafterbreak}{\hbox{\char`\?}}}%
            \lccode`\~`\!\lowercase{\def~}{\discretionary{\hbox{\char`\!}}{\Wrappedafterbreak}{\hbox{\char`\!}}}%
            \lccode`\~`\/\lowercase{\def~}{\discretionary{\hbox{\char`\/}}{\Wrappedafterbreak}{\hbox{\char`\/}}}%
            \catcode`\.\active
            \catcode`\,\active
            \catcode`\;\active
            \catcode`\:\active
            \catcode`\?\active
            \catcode`\!\active
            \catcode`\/\active
            \lccode`\~`\~
        }
    \makeatother

    \let\OriginalVerbatim=\Verbatim
    \makeatletter
    \renewcommand{\Verbatim}[1][1]{%
        %\parskip\z@skip
        \sbox\Wrappedcontinuationbox {\Wrappedcontinuationsymbol}%
        \sbox\Wrappedvisiblespacebox {\FV@SetupFont\Wrappedvisiblespace}%
        \def\FancyVerbFormatLine ##1{\hsize\linewidth
            \vtop{\raggedright\hyphenpenalty\z@\exhyphenpenalty\z@
                \doublehyphendemerits\z@\finalhyphendemerits\z@
                \strut ##1\strut}%
        }%
        % If the linebreak is at a space, the latter will be displayed as visible
        % space at end of first line, and a continuation symbol starts next line.
        % Stretch/shrink are however usually zero for typewriter font.
        \def\FV@Space {%
            \nobreak\hskip\z@ plus\fontdimen3\font minus\fontdimen4\font
            \discretionary{\copy\Wrappedvisiblespacebox}{\Wrappedafterbreak}
            {\kern\fontdimen2\font}%
        }%

        % Allow breaks at special characters using \PYG... macros.
        \Wrappedbreaksatspecials
        % Breaks at punctuation characters . , ; ? ! and / need catcode=\active
        \OriginalVerbatim[#1,codes*=\Wrappedbreaksatpunct]%
    }
    \makeatother

    % Exact colors from NB
    \definecolor{incolor}{HTML}{303F9F}
    \definecolor{outcolor}{HTML}{D84315}
    \definecolor{cellborder}{HTML}{CFCFCF}
    \definecolor{cellbackground}{HTML}{F7F7F7}

    % prompt
    \makeatletter
    \newcommand{\boxspacing}{\kern\kvtcb@left@rule\kern\kvtcb@boxsep}
    \makeatother
    \newcommand{\prompt}[4]{
        {\ttfamily\llap{{\color{#2}[#3]:\hspace{3pt}#4}}\vspace{-\baselineskip}}
    }
    

    
    % Prevent overflowing lines due to hard-to-break entities
    \sloppy
    % Setup hyperref package
    \hypersetup{
      breaklinks=true,  % so long urls are correctly broken across lines
      colorlinks=true,
      urlcolor=urlcolor,
      linkcolor=linkcolor,
      citecolor=citecolor,
      }
    % Slightly bigger margins than the latex defaults
    
    \geometry{verbose,tmargin=1in,bmargin=1in,lmargin=1in,rmargin=1in}
    
\usepackage{listings}
\lstset{
  basicstyle=\ttfamily,
  breaklines=true,
  aboveskip=10pt, % adjust the space above the code block
  belowskip=5pt, % adjust the space below the code block
}



\begin{document}
    
    \maketitle
    

    
    \hypertarget{vorbereitung}{%
\section{Vorbereitung}\label{vorbereitung}}

    \hypertarget{einleitung}{%
\subsection{Einleitung}\label{einleitung}}

    Unser Datensatz mit dem Titel \textbf{``Historische Kraftstoffpreisdaten
in Deutschland''} lässt sich unter
https://dev.azure.com/tankerkoenig/\_git/tankerkoenig-data finden und
ist in \textbf{Stations} und \textbf{Prices} aufgeteilt. Stations listet
jede aktive Tankstelle in Deutschland auf, wohingegen in Prices alle
Preisänderungen von Diesel, Super 95 (E5) und Super E10 von Juni 2014
bis jetzt enthalten sind. Anfänglich stellten wir uns die Frage, wie es
möglich sei die Fülle dieser Daten so akkurat zu erfassen. Unsere
Hintergrundrecherche ergab, dass jede öffentliche Tankstelle seit dem
31. August 2013 Preisänderungen von Super 95, Super E10 und Diesel in
Echtzeit an die sog. \textbf{Markttransparenzstelle für Kraftstoffe} des
Bundeskartellamtes melden muss. Auf Grund der Tatsache, dass das
Bundeskartellamt die gewonnenen Daten ausschließlich lizensierten
Spritpreis-Informationsdiensten über eine API-Schnittstelle zur
Verfügung stellt, beziehen wir unsere Daten aus dem Archiv des Tankerkönigs, welches mit dem Befehl

\begin{lstlisting}
git clone https://tankerkoenig@dev.azure.com/tankerkoenig/tankerkoenig-data/_git/tankerkoenig-data
\end{lstlisting}

heruntergeladen werden kann. Die dort archivierten Daten werden seit
Juni 2014 tagesweise archiviert und jeweils am Folgetag als
Datenbank-Dump in Form einer CSV-Datei bereitgestellt. Auf diese Weise
umfasst das Archiv mehr als 3.000 CSV-Dateien mit durchschnittlich ca.
350.000 Einträgen, so dass man auf eine Gesamtanzahl von über 1
Milliarde Datenpunkten kommt. Welche Problematiken diese riesige
Datenmenge auslöste und wie wir diese Probleme letztendlich lösen
konnten, erklären wir im weiteren Verlauf.

Beschäftigen wir uns nun aber zuerst mit dem eigentlichen Ablauf dieser
Datenvisualisierung. Anfänglich werden wir uns mit dem Datensatz
vertraut machen, die Bedeutungen der verschiedenen Spalten klären und
etwas Data Cleaning betreiben. Daraufhin werden wir ein Verständnis über
die größten Tankstellenmarken und deren Marktanteile aufbauen und uns
anschließend die geografischen Unterschiede anzeigen lassen. Hier werden
wir speziell auf die, nach wie vor bestehenden, Unterschiede zwischen
``Ost'' und ``West'' eingehen. Abschließend skizzieren wir die
historische Preisentwicklung und werden die aktuellen und
herausstechende Preisänderungen seit Beginn des Krieges in der Ukraine
hervorheben.

    \hypertarget{library-imports}{%
\subsection{Library Imports}\label{library-imports}}

    Damit wir unsere Daten fachgerecht bereinigen, präparieren und
schlussendlich visualisieren können, importieren wir folgende, in der
Datenanalyse- und visualisierung weit verbreitete, Biblotheken. Diese
werden von uns zur Visualisierung und Aufbereitung der Daten benutzt und
können teilweise als Pendant zu der in R häufig verwendeten Biblothek
\emph{ggplot2} angesehen werden:

\begin{enumerate}
\def\labelenumi{\arabic{enumi}.}
\item
  \textbf{pandas}: Eine Python-Bibliothek, die speziell für die
  Datenmanipulation und -analyse entwickelt wurde. Sie ermöglicht es,
  Datenframes zu erstellen und bietet eine Vielzahl nützlicher
  Funktionen zur Datenbereinigung und -verarbeitung.
\item
  \textbf{matplotlib.pyplot}: Eine der bekanntesten Python-Bibliotheken
  zur Erstellung von Diagrammen und Visualisierungen. Mit pyplot können
  Sie eine Vielzahl von Diagrammtypen erstellen, darunter
  Balkendiagramme, Liniendiagramme und Histogramme.
\item
  \textbf{seaborn}: Eine Python-Bibliothek zur Erstellung ansprechender
  Diagramme und Visualisierungen. Sie baut auf matplotlib auf und bietet
  einfachere Methoden zur Erstellung von Diagrammen.
\item
  \textbf{os}: Eine Python-Bibliothek, die Methoden zum Arbeiten mit dem
  Betriebssystem bereitstellt, wie zum Beispiel zum Navigieren im
  Dateisystem oder zum Erstellen von Verzeichnissen.
\item
  \textbf{PolygonPatch}: Die Klasse PolygonPatch wird in der Regel in
  Verbindung mit anderen Bibliotheken wie matplotlib oder geopandas
  verwendet, um visuelle Darstellungen von Polygonen in Diagrammen oder
  Karten zu erstellen. Sie bietet eine bequeme Möglichkeit,
  Polygon-Patches mit spezifischen Attributen wie Farben, Kantenbreiten
  und Transparenz hinzuzufügen, um die visuelle Präsentation von
  räumlichen Daten zu verbessern.
\item
  \textbf{shapefile}: Eine Python-Bibliothek zur Arbeit mit Shapefiles,
  einem geografischen Vektordatendateiformat. Sie ermöglicht das Lesen,
  Schreiben und Manipulieren von Shapefiles für die Analyse und
  Visualisierung von geografischen Daten in Python.
\item
  \textbf{datetime}: Ein Modul in Python, das Funktionen und Klassen zur
  Arbeit mit Datum und Zeit bereitstellt. Es ermöglicht das Erstellen,
  Manipulieren und Formatieren von Datum- und Zeitobjekten. Mit dem
  datetime-Modul können Berechnungen und Vergleiche zwischen
  verschiedenen Zeitpunkten durchgeführt werden.
\item
  \textbf{dateutil.relativedelta}: Eine Funktion aus dem dateutil-Paket,
  das auf das datetime-Modul aufbaut. Die dateutil.relativedelta
  ermöglicht es, Zeitdifferenzen zwischen zwei Zeitpunkten in einem
  flexiblen und intuitiven Format zu berechnen. Es erlaubt die Angabe
  von relativen Änderungen wie ``2 Monate vorwärts'' oder ``1 Jahr und 3
  Tage rückwärts'' und ermöglicht es, solche Zeitdifferenzen präzise zu
  berechnen und auf Datum- oder Zeitobjekte anzuwenden.
\end{enumerate}

    \begin{tcolorbox}[breakable, size=fbox, boxrule=1pt, pad at break*=1mm,colback=cellbackground, colframe=cellborder]
\prompt{In}{incolor}{1}{\boxspacing}
\begin{Verbatim}[commandchars=\\\{\}]
\PY{k+kn}{import} \PY{n+nn}{pandas} \PY{k}{as} \PY{n+nn}{pd}
\PY{k+kn}{import} \PY{n+nn}{matplotlib}\PY{n+nn}{.}\PY{n+nn}{pyplot} \PY{k}{as} \PY{n+nn}{plt}
\PY{k+kn}{import} \PY{n+nn}{seaborn} \PY{k}{as} \PY{n+nn}{sns}
\PY{k+kn}{import} \PY{n+nn}{os}
\PY{k+kn}{from}   \PY{n+nn}{descartes} \PY{k+kn}{import} \PY{n}{PolygonPatch}
\PY{k+kn}{import} \PY{n+nn}{shapefile}
\PY{k+kn}{from} \PY{n+nn}{datetime} \PY{k+kn}{import} \PY{n}{datetime}
\PY{k+kn}{from} \PY{n+nn}{dateutil}\PY{n+nn}{.}\PY{n+nn}{relativedelta} \PY{k+kn}{import} \PY{n}{relativedelta}
\end{Verbatim}
\end{tcolorbox}

    \hypertarget{data-imports}{%
\subsection{Data Imports}\label{data-imports}}

    \hypertarget{preprocessing}{%
\subsubsection{Preprocessing}\label{preprocessing}}

    Wie so oft geht das ganze nicht ohne kleine aber sehr hilfreichen
Tricks, in unserem Fall \textbf{Preprocessing}. Preprocessing bezeichnet
die Vorverarbeitung von Rohdaten, um sie für die weitere Verarbeitung
und Analyse vorzubereiten. Da die Prices-CSV-Dateien insgesamt eine
Größe von über 50 GB haben und jegliche pro Tankstelle anfallende
Spritpreisaktualisierungen in Deutschland seit 2014 enthalten, mussten
wir die Größe reduzieren, um sie effizient verarbeiten zu können. Wir
entschieden uns, die Daten nach station\_id, Jahr und Monat zu
gruppieren und die Durchschnittspreise für die Kraftstoffarten Diesel,
Super 95 (E5) und Super E10 einmal pro Monat anstelle pro Minute zu
berechnen und in neuen CSV-Dateien zu speichern. Dadurch konnten wir die
Gesamtgröße von über 50 GB auf etwa 90 MB reduzieren. Die Auslagerung
der Aggregation in eine separate Funktion dient der besseren Performance
und besseren Lesbarkeit~des~Codes. Die detaillierte Beschreibung des
Codes würde hier aber zu weit gehen, weshalb unser Code hierzu bei
Interesse im Anhang zu finden ist.

    \hypertarget{dataframes}{%
\subsubsection{DataFrames}\label{dataframes}}

    Nach dem uns nun die aggregierten Daten vorliegen werden die
entstandenen CSV-Dateien anhand des folgenden Codes eingelesen und in
verschiedenen DataFrames gespeichert. Dies ist ein wichtiger Schritt, um
eine breitere Datenbasis für zukünftige Analysen zu erhalten und
sicherzustellen, dass die Daten korrekt interpretiert werden können.
Zusätzlich werden auch fehlerhafte CSV-Dateien gefiltert und
übersprungen, um mögliche Fehler in den Daten zu vermeiden.

Um eine einheitliche und korrekte Verarbeitung der Daten zu
gewährleisten, wird die Datum-Spalte in das Datumformat umgewandelt.
Durch das Einlesen und Zusammenführen der Daten aus verschiedenen
CSV-Dateien in verschiedenen DataFrames wird eine bessere Analyse und
Verarbeitung der Daten ermöglicht. Dies ist insbesondere von Vorteil, da
die CSV-Dateien unterschiedliche Themenbereiche abdecken.

Der Code durchläuft mittels Schleife den Ordner, in dem die CSV-Dateien
gespeichert sind, liest diese ein und speichert die darin enthaltenen
Daten in verschiedenen DataFrames. Für den Monat März 2023 wird ein
leeres DataFrame ``march\_2023'' initialisiert, um die Daten zu
speichern. Für den Ordner ``timeseries'' wird ebenfalls ein leeres
DataFrame ``timeseries'' erstellt, um die Daten aus den CSV-Dateien zu
speichern.

Nach dem Einlesen und Verarbeiten der CSV-Dateien werden die
Informationen zu den verschiedenen Stationen, von denen die Daten
stammen, in das DataFrame ``stations'' geladen.

    \begin{tcolorbox}[breakable, size=fbox, boxrule=1pt, pad at break*=1mm,colback=cellbackground, colframe=cellborder]
\prompt{In}{incolor}{2}{\boxspacing}
\begin{Verbatim}[commandchars=\\\{\}]
\PY{c+c1}{\PYZsh{} path + df to store}
\PY{n}{folder\PYZus{}path} \PY{o}{=} \PY{l+s+s1}{\PYZsq{}}\PY{l+s+s1}{04}\PY{l+s+s1}{\PYZsq{}}
\PY{n}{april\PYZus{}2023} \PY{o}{=} \PY{n}{pd}\PY{o}{.}\PY{n}{DataFrame}\PY{p}{(}\PY{p}{)}

\PY{c+c1}{\PYZsh{} looping through folder + append + date fixing}
\PY{k}{for} \PY{n}{filename} \PY{o+ow}{in} \PY{n}{os}\PY{o}{.}\PY{n}{listdir}\PY{p}{(}\PY{n}{folder\PYZus{}path}\PY{p}{)}\PY{p}{:}
    \PY{k}{if} \PY{n}{filename}\PY{o}{.}\PY{n}{endswith}\PY{p}{(}\PY{l+s+s1}{\PYZsq{}}\PY{l+s+s1}{.csv}\PY{l+s+s1}{\PYZsq{}}\PY{p}{)}\PY{p}{:}
        \PY{n}{file\PYZus{}path} \PY{o}{=} \PY{n}{os}\PY{o}{.}\PY{n}{path}\PY{o}{.}\PY{n}{join}\PY{p}{(}\PY{n}{folder\PYZus{}path}\PY{p}{,} \PY{n}{filename}\PY{p}{)}
        \PY{k}{try}\PY{p}{:}
            \PY{n}{temp\PYZus{}df} \PY{o}{=} \PY{n}{pd}\PY{o}{.}\PY{n}{read\PYZus{}csv}\PY{p}{(}\PY{n}{file\PYZus{}path}\PY{p}{)}
            \PY{n}{temp\PYZus{}df}\PY{p}{[}\PY{l+s+s1}{\PYZsq{}}\PY{l+s+s1}{date}\PY{l+s+s1}{\PYZsq{}}\PY{p}{]} \PY{o}{=} \PY{n}{pd}\PY{o}{.}\PY{n}{to\PYZus{}datetime}\PY{p}{(}\PY{n}{temp\PYZus{}df}\PY{p}{[}\PY{l+s+s1}{\PYZsq{}}\PY{l+s+s1}{date}\PY{l+s+s1}{\PYZsq{}}\PY{p}{]}\PY{p}{,} \PY{n+nb}{format}\PY{o}{=}\PY{l+s+s1}{\PYZsq{}}\PY{l+s+s1}{\PYZpc{}}\PY{l+s+s1}{Y\PYZhy{}}\PY{l+s+s1}{\PYZpc{}}\PY{l+s+s1}{m\PYZhy{}}\PY{l+s+si}{\PYZpc{}d}\PY{l+s+s1}{\PYZsq{}}\PY{p}{)}\PY{o}{.}\PY{n}{dt}\PY{o}{.}\PY{n}{date}
            \PY{n}{april\PYZus{}2023} \PY{o}{=} \PY{n}{april\PYZus{}2023}\PY{o}{.}\PY{n}{append}\PY{p}{(}\PY{n}{temp\PYZus{}df}\PY{p}{)}
        \PY{k}{except} \PY{n+ne}{Exception}\PY{p}{:}
            \PY{c+c1}{\PYZsh{} if something is wrong in the csv file}
            \PY{n+nb}{print}\PY{p}{(}\PY{l+s+sa}{f}\PY{l+s+s2}{\PYZdq{}}\PY{l+s+s2}{Error converting date column in }\PY{l+s+si}{\PYZob{}}\PY{n}{filename}\PY{l+s+si}{\PYZcb{}}\PY{l+s+s2}{. Skipping file...}\PY{l+s+s2}{\PYZdq{}}\PY{p}{)}

\PY{c+c1}{\PYZsh{} path + df to store}
\PY{n}{folder\PYZus{}path2} \PY{o}{=} \PY{l+s+s1}{\PYZsq{}}\PY{l+s+s1}{timeseries}\PY{l+s+s1}{\PYZsq{}}
\PY{n}{timeseries} \PY{o}{=} \PY{n}{pd}\PY{o}{.}\PY{n}{DataFrame}\PY{p}{(}\PY{p}{)}

\PY{c+c1}{\PYZsh{} looping through folder + append + date fixing}
\PY{k}{for} \PY{n}{filename} \PY{o+ow}{in} \PY{n}{os}\PY{o}{.}\PY{n}{listdir}\PY{p}{(}\PY{n}{folder\PYZus{}path2}\PY{p}{)}\PY{p}{:}
    \PY{k}{if} \PY{n}{filename}\PY{o}{.}\PY{n}{endswith}\PY{p}{(}\PY{l+s+s1}{\PYZsq{}}\PY{l+s+s1}{.csv}\PY{l+s+s1}{\PYZsq{}}\PY{p}{)}\PY{p}{:}
        \PY{n}{file\PYZus{}path} \PY{o}{=} \PY{n}{os}\PY{o}{.}\PY{n}{path}\PY{o}{.}\PY{n}{join}\PY{p}{(}\PY{n}{folder\PYZus{}path2}\PY{p}{,} \PY{n}{filename}\PY{p}{)}
        \PY{n}{temp\PYZus{}df} \PY{o}{=} \PY{n}{pd}\PY{o}{.}\PY{n}{read\PYZus{}csv}\PY{p}{(}\PY{n}{file\PYZus{}path}\PY{p}{)}
        \PY{k}{try}\PY{p}{:}
            \PY{n}{timeseries} \PY{o}{=} \PY{n}{timeseries}\PY{o}{.}\PY{n}{append}\PY{p}{(}\PY{n}{temp\PYZus{}df}\PY{p}{)}
        \PY{k}{except} \PY{n+ne}{Exception}\PY{p}{:}
            \PY{c+c1}{\PYZsh{} if something is wrong in the csv file}
            \PY{n+nb}{print}\PY{p}{(}\PY{l+s+sa}{f}\PY{l+s+s2}{\PYZdq{}}\PY{l+s+s2}{Error converting date column in }\PY{l+s+si}{\PYZob{}}\PY{n}{filename}\PY{l+s+si}{\PYZcb{}}\PY{l+s+s2}{. Skipping file...}\PY{l+s+s2}{\PYZdq{}}\PY{p}{)}

\PY{c+c1}{\PYZsh{} read stations}
\PY{n}{stations} \PY{o}{=} \PY{n}{pd}\PY{o}{.}\PY{n}{read\PYZus{}csv}\PY{p}{(}\PY{l+s+s1}{\PYZsq{}}\PY{l+s+s1}{stations.csv}\PY{l+s+s1}{\PYZsq{}}\PY{p}{)}
\end{Verbatim}
\end{tcolorbox}
    \hypertarget{variablenbeschreibung}{%
\subsection{Variablenbeschreibung}\label{variablenbeschreibung}}

   Die vorliegenden Datensätze bestehen aus folgenden Variablen: Die Variable ``station\_uuid'' steht für ``Universally Unique
Identifier'' und ist eine 128-Bit-Zahl, die zur Identifizierung von
Objekten in Computersystemen verwendet wird und wurde von der Open
Software Foundation (OSF) als Teil des Distributed Computing Environment
(DCE) standardisiert und ist jetzt in RFC 4122 geregelt. ``city'' ist
der Name der Stadt, in der die Tankstelle steht. In ``brand'' ist die
Marke der Tankstelle genannt, wie z.B. ``ARAL'' oder ``Shell''.
``post\_code'' ist die fünfstellige Postleitzahl des Ortes der Tankstelle. ``latitude'' ist der geographische Breitengrad in
Grad, auf der sich die Tankstelle befindet. ``longitude'' ist der
geographische Längengrad in Grad, auf der sich die Tankstelle befindet,
welcher mit ``laditude'' die Koordinaten der Tankstelle bildet.
Die Variablen ``diesel'', ``e10'' und ``e5'' des ``prices''-Datensatzes
beinhalten den Preis des jeweiligen Kraftstoffes auf die dritte
Nachkommastelle genau in Euro, z.B. ``1.689'' Euro. An welchem Tag die
Daten erfasst wurden, wird in der Variable ``date'' dokumentiert, wobei
das Datum das Format YYYY-MM-DD hat. Die Preise einer Zeile wurden
gleichzeitig erfasst. Auch in Stations ist die die Spalte ``uuid''
enthalten, so dass Tankstellendaten und Preise zusammengeführt werden
können.

\newpage
    \hypertarget{data-cleaning}{%
\subsection{Data Cleaning}\label{data-cleaning}}

    In der Zeitreihenanalyse ist die Datenbereinigung ein wichtiger Schritt,
um sicherzustellen, dass die verwendeten Daten korrekt und zuverlässig
sind. Wir führen verschiedene Schritte durch, um sicherzustellen, dass
die Daten sauber sind. Zunächst erstellen wir eine neue Spalte mit
Datumsinformationen, um die Handhabung der zeitabhängigen Daten zu
erleichtern. Dann filtern wir nur die für unsere Analyse relevanten
Spalten in den Datenframes und entfernen Zeilen mit fehlenden oder
unzuverlässigen Daten. Dadurch stellen wir sicher, dass unsere Analyse
auf vertrauenswürdigen Daten basiert und aussagekräftige Ergebnisse
liefert.

    \begin{tcolorbox}[breakable, size=fbox, boxrule=1pt, pad at break*=1mm,colback=cellbackground, colframe=cellborder]
\prompt{In}{incolor}{3}{\boxspacing}
\begin{Verbatim}[commandchars=\\\{\}]
\PY{c+c1}{\PYZsh{}correcting date}
\PY{n}{timeseries}\PY{p}{[}\PY{l+s+s1}{\PYZsq{}}\PY{l+s+s1}{date}\PY{l+s+s1}{\PYZsq{}}\PY{p}{]} \PY{o}{=} \PY{n}{timeseries}\PY{p}{[}\PY{l+s+s1}{\PYZsq{}}\PY{l+s+s1}{year}\PY{l+s+s1}{\PYZsq{}}\PY{p}{]}\PY{o}{.}\PY{n}{astype}\PY{p}{(}\PY{n+nb}{str}\PY{p}{)} \PY{o}{+} \PY{l+s+s1}{\PYZsq{}}\PY{l+s+s1}{ }\PY{l+s+s1}{\PYZsq{}} \PY{o}{+} \PY{n}{timeseries}\PY{p}{[}\PY{l+s+s1}{\PYZsq{}}\PY{l+s+s1}{month}\PY{l+s+s1}{\PYZsq{}}\PY{p}{]}\PY{o}{.}\PY{n}{apply}\PY{p}{(}\PY{k}{lambda} \PY{n}{x}\PY{p}{:} \PY{l+s+s1}{\PYZsq{}}\PY{l+s+si}{\PYZob{}:02\PYZcb{}}\PY{l+s+s1}{\PYZsq{}}\PY{o}{.}\PY{n}{format}\PY{p}{(}\PY{n}{x}\PY{p}{)}\PY{p}{)}
\PY{n}{timeseries}\PY{p}{[}\PY{l+s+s1}{\PYZsq{}}\PY{l+s+s1}{date}\PY{l+s+s1}{\PYZsq{}}\PY{p}{]} \PY{o}{=} \PY{n}{pd}\PY{o}{.}\PY{n}{to\PYZus{}datetime}\PY{p}{(}\PY{n}{timeseries}\PY{p}{[}\PY{l+s+s1}{\PYZsq{}}\PY{l+s+s1}{date}\PY{l+s+s1}{\PYZsq{}}\PY{p}{]}\PY{p}{,} \PY{n+nb}{format}\PY{o}{=}\PY{l+s+s1}{\PYZsq{}}\PY{l+s+s1}{\PYZpc{}}\PY{l+s+s1}{Y }\PY{l+s+s1}{\PYZpc{}}\PY{l+s+s1}{m}\PY{l+s+s1}{\PYZsq{}}\PY{p}{)}

\PY{c+c1}{\PYZsh{}\PYZsh{}\PYZsh{} filter for relevant variables}
\PY{n}{stations} \PY{o}{=} \PY{n}{stations}\PY{p}{[}\PY{p}{[}\PY{l+s+s1}{\PYZsq{}}\PY{l+s+s1}{uuid}\PY{l+s+s1}{\PYZsq{}}\PY{p}{,} \PY{l+s+s1}{\PYZsq{}}\PY{l+s+s1}{city}\PY{l+s+s1}{\PYZsq{}}\PY{p}{,} \PY{l+s+s1}{\PYZsq{}}\PY{l+s+s1}{brand}\PY{l+s+s1}{\PYZsq{}}\PY{p}{,} \PY{l+s+s1}{\PYZsq{}}\PY{l+s+s1}{post\PYZus{}code}\PY{l+s+s1}{\PYZsq{}}\PY{p}{,} \PY{l+s+s1}{\PYZsq{}}\PY{l+s+s1}{latitude}\PY{l+s+s1}{\PYZsq{}}\PY{p}{,} \PY{l+s+s1}{\PYZsq{}}\PY{l+s+s1}{longitude}\PY{l+s+s1}{\PYZsq{}}\PY{p}{]}\PY{p}{]}
\PY{n}{timeseries} \PY{o}{=} \PY{n}{timeseries}\PY{p}{[}\PY{p}{[}\PY{l+s+s1}{\PYZsq{}}\PY{l+s+s1}{station\PYZus{}uuid}\PY{l+s+s1}{\PYZsq{}}\PY{p}{,} \PY{l+s+s1}{\PYZsq{}}\PY{l+s+s1}{date}\PY{l+s+s1}{\PYZsq{}}\PY{p}{,} \PY{l+s+s1}{\PYZsq{}}\PY{l+s+s1}{diesel}\PY{l+s+s1}{\PYZsq{}}\PY{p}{,} \PY{l+s+s1}{\PYZsq{}}\PY{l+s+s1}{e5}\PY{l+s+s1}{\PYZsq{}}\PY{p}{,} \PY{l+s+s1}{\PYZsq{}}\PY{l+s+s1}{e10}\PY{l+s+s1}{\PYZsq{}}\PY{p}{]}\PY{p}{]}
\PY{n}{april\PYZus{}2023} \PY{o}{=} \PY{n}{april\PYZus{}2023}\PY{p}{[}\PY{p}{[}\PY{l+s+s1}{\PYZsq{}}\PY{l+s+s1}{station\PYZus{}uuid}\PY{l+s+s1}{\PYZsq{}}\PY{p}{,} \PY{l+s+s1}{\PYZsq{}}\PY{l+s+s1}{date}\PY{l+s+s1}{\PYZsq{}}\PY{p}{,} \PY{l+s+s1}{\PYZsq{}}\PY{l+s+s1}{diesel}\PY{l+s+s1}{\PYZsq{}}\PY{p}{,} \PY{l+s+s1}{\PYZsq{}}\PY{l+s+s1}{e5}\PY{l+s+s1}{\PYZsq{}}\PY{p}{,} \PY{l+s+s1}{\PYZsq{}}\PY{l+s+s1}{e10}\PY{l+s+s1}{\PYZsq{}}\PY{p}{]}\PY{p}{]}

\PY{c+c1}{\PYZsh{}\PYZsh{}\PYZsh{} renaming uuid}
\PY{n}{stations} \PY{o}{=} \PY{n}{stations}\PY{o}{.}\PY{n}{rename}\PY{p}{(} \PY{n}{columns}\PY{o}{=} \PY{p}{\PYZob{}}\PY{l+s+s1}{\PYZsq{}}\PY{l+s+s1}{uuid}\PY{l+s+s1}{\PYZsq{}} \PY{p}{:} \PY{l+s+s1}{\PYZsq{}}\PY{l+s+s1}{station\PYZus{}uuid}\PY{l+s+s1}{\PYZsq{}}\PY{p}{\PYZcb{}}\PY{p}{)}

\PY{c+c1}{\PYZsh{} drop NaN values and drop rows where diesel or e10 or e5 are 0}
\PY{n}{stations}\PY{o}{.}\PY{n}{dropna}\PY{p}{(}\PY{n}{inplace}\PY{o}{=}\PY{k+kc}{True}\PY{p}{)}
\PY{n}{timeseries}\PY{o}{.}\PY{n}{dropna}\PY{p}{(}\PY{n}{inplace}\PY{o}{=}\PY{k+kc}{True}\PY{p}{)}
\PY{n}{april\PYZus{}2023}\PY{o}{.}\PY{n}{dropna}\PY{p}{(}\PY{n}{inplace}\PY{o}{=}\PY{k+kc}{True}\PY{p}{)}
\PY{n}{timeseries} \PY{o}{=} \PY{n}{timeseries}\PY{o}{.}\PY{n}{loc}\PY{p}{[}\PY{p}{(}\PY{n}{timeseries}\PY{o}{.}\PY{n}{diesel} \PY{o}{!=} \PY{l+m+mi}{0}\PY{p}{)} \PY{o}{\PYZam{}} \PY{p}{(}\PY{n}{timeseries}\PY{o}{.}\PY{n}{e10} \PY{o}{!=} \PY{l+m+mi}{0}\PY{p}{)} \PY{o}{\PYZam{}} \PY{p}{(}\PY{n}{timeseries}\PY{o}{.}\PY{n}{e5} \PY{o}{!=} \PY{l+m+mi}{0}\PY{p}{)}\PY{p}{]}
\PY{n}{march\PYZus{}2023} \PY{o}{=} \PY{n}{april\PYZus{}2023}\PY{o}{.}\PY{n}{loc}\PY{p}{[}\PY{p}{(}\PY{n}{april\PYZus{}2023}\PY{o}{.}\PY{n}{diesel} \PY{o}{!=} \PY{l+m+mi}{0}\PY{p}{)} \PY{o}{\PYZam{}} \PY{p}{(}\PY{n}{april\PYZus{}2023}\PY{o}{.}\PY{n}{e10} \PY{o}{!=} \PY{l+m+mi}{0}\PY{p}{)} \PY{o}{\PYZam{}} \PY{p}{(}\PY{n}{april\PYZus{}2023}\PY{o}{.}\PY{n}{e5} \PY{o}{!=} \PY{l+m+mi}{0}\PY{p}{)}\PY{p}{]}
\end{Verbatim}
\end{tcolorbox}



\newpage
    \hypertarget{plots}{%
\section{Plots}\label{plots}}


    \hypertarget{verteilung-der-deutschen-marktteilnehmer-in-der-tankstellenbranche}{%
\subsection{Verteilung der deutschen Marktteilnehmer in der
Tankstellenbranche}\label{verteilung-der-deutschen-marktteilnehmer-in-der-tankstellenbranche}}

    Mit dem folgenden Code stellen wir einen Überblick über die in
Deutschland vertretenen Tankstellenmarken und deren Marktanteile dar. Hierbei beschränken wir uns aus Gründen der besseren Übersichtlichkeit
auf die Top 5. 


    \begin{tcolorbox}[breakable, size=fbox, boxrule=1pt, pad at break*=1mm,colback=cellbackground, colframe=cellborder]
\prompt{In}{incolor}{4}{\boxspacing}
\begin{Verbatim}[commandchars=\\\{\}]
\PY{c+c1}{\PYZsh{} for same size etc}
\PY{n}{title\PYZus{}fontsize} \PY{o}{=} \PY{l+m+mi}{20}
\PY{n}{width} \PY{o}{=} \PY{l+m+mi}{14}

\PY{c+c1}{\PYZsh{} Load the dataset}
\PY{n}{brand\PYZus{}stations} \PY{o}{=} \PY{n}{stations}

\PY{c+c1}{\PYZsh{} Filter for top 5 brands}
\PY{n}{top\PYZus{}brands} \PY{o}{=} \PY{n}{brand\PYZus{}stations}\PY{o}{.}\PY{n}{groupby}\PY{p}{(}\PY{l+s+s1}{\PYZsq{}}\PY{l+s+s1}{brand}\PY{l+s+s1}{\PYZsq{}}\PY{p}{)}\PY{p}{[}\PY{l+s+s1}{\PYZsq{}}\PY{l+s+s1}{brand}\PY{l+s+s1}{\PYZsq{}}\PY{p}{]}\PY{o}{.}\PY{n}{count}\PY{p}{(}\PY{p}{)}\PY{o}{.}\PY{n}{reset\PYZus{}index}\PY{p}{(}\PY{n}{name}\PY{o}{=}\PY{l+s+s1}{\PYZsq{}}\PY{l+s+s1}{count}\PY{l+s+s1}{\PYZsq{}}\PY{p}{)}\PY{o}{.}\PY{n}{nlargest}\PY{p}{(}\PY{l+m+mi}{5}\PY{p}{,} \PY{l+s+s1}{\PYZsq{}}\PY{l+s+s1}{count}\PY{l+s+s1}{\PYZsq{}}\PY{p}{)}
\PY{n}{top\PYZus{}brands\PYZus{}list} \PY{o}{=} \PY{n}{top\PYZus{}brands}\PY{p}{[}\PY{l+s+s1}{\PYZsq{}}\PY{l+s+s1}{brand}\PY{l+s+s1}{\PYZsq{}}\PY{p}{]}\PY{o}{.}\PY{n}{unique}\PY{p}{(}\PY{p}{)}
\PY{n}{rest\PYZus{}brands} \PY{o}{=} \PY{n}{brand\PYZus{}stations}\PY{o}{.}\PY{n}{loc}\PY{p}{[}\PY{o}{\PYZti{}}\PY{n}{brand\PYZus{}stations}\PY{p}{[}\PY{l+s+s1}{\PYZsq{}}\PY{l+s+s1}{brand}\PY{l+s+s1}{\PYZsq{}}\PY{p}{]}\PY{o}{.}\PY{n}{isin}\PY{p}{(}\PY{n}{top\PYZus{}brands\PYZus{}list}\PY{p}{)}\PY{p}{]}\PY{o}{.}\PY{n}{agg}\PY{p}{(}\PY{p}{\PYZob{}}\PY{l+s+s1}{\PYZsq{}}\PY{l+s+s1}{brand}\PY{l+s+s1}{\PYZsq{}}\PY{p}{:} \PY{l+s+s1}{\PYZsq{}}\PY{l+s+s1}{count}\PY{l+s+s1}{\PYZsq{}}\PY{p}{\PYZcb{}}\PY{p}{)}\PY{o}{.}\PY{n}{reset\PYZus{}index}\PY{p}{(}\PY{p}{)}
\PY{n}{rest\PYZus{}brands} \PY{o}{=} \PY{n}{rest\PYZus{}brands}\PY{o}{.}\PY{n}{rename}\PY{p}{(}\PY{n}{columns}\PY{o}{=}\PY{p}{\PYZob{}}\PY{l+s+s1}{\PYZsq{}}\PY{l+s+s1}{index}\PY{l+s+s1}{\PYZsq{}}\PY{p}{:} \PY{l+s+s1}{\PYZsq{}}\PY{l+s+s1}{brand}\PY{l+s+s1}{\PYZsq{}}\PY{p}{,} \PY{l+m+mi}{0} \PY{p}{:} \PY{l+s+s1}{\PYZsq{}}\PY{l+s+s1}{count}\PY{l+s+s1}{\PYZsq{}}\PY{p}{\PYZcb{}}\PY{p}{)}
\PY{n}{rest\PYZus{}brands}\PY{p}{[}\PY{l+s+s1}{\PYZsq{}}\PY{l+s+s1}{brand}\PY{l+s+s1}{\PYZsq{}}\PY{p}{]} \PY{o}{=} \PY{n}{rest\PYZus{}brands}\PY{p}{[}\PY{l+s+s1}{\PYZsq{}}\PY{l+s+s1}{brand}\PY{l+s+s1}{\PYZsq{}}\PY{p}{]}\PY{o}{.}\PY{n}{replace}\PY{p}{(}\PY{l+s+s1}{\PYZsq{}}\PY{l+s+s1}{brand}\PY{l+s+s1}{\PYZsq{}}\PY{p}{,} \PY{l+s+s1}{\PYZsq{}}\PY{l+s+s1}{Rest}\PY{l+s+s1}{\PYZsq{}}\PY{p}{)}

\PY{c+c1}{\PYZsh{} colors for brands}
\PY{n}{custom\PYZus{}palette} \PY{o}{=} \PY{p}{[}\PY{l+s+s1}{\PYZsq{}}\PY{l+s+s1}{\PYZsh{}1670B9}\PY{l+s+s1}{\PYZsq{}}\PY{p}{,} \PY{l+s+s1}{\PYZsq{}}\PY{l+s+s1}{\PYZsh{}FFD400}\PY{l+s+s1}{\PYZsq{}}\PY{p}{,} \PY{l+s+s1}{\PYZsq{}}\PY{l+s+s1}{\PYZsh{}DA0013}\PY{l+s+s1}{\PYZsq{}}\PY{p}{,} \PY{l+s+s1}{\PYZsq{}}\PY{l+s+s1}{\PYZsh{}A526AB}\PY{l+s+s1}{\PYZsq{}}\PY{p}{,} \PY{l+s+s1}{\PYZsq{}}\PY{l+s+s1}{\PYZsh{}109A76}\PY{l+s+s1}{\PYZsq{}}\PY{p}{,} \PY{l+s+s1}{\PYZsq{}}\PY{l+s+s1}{\PYZsh{}777777}\PY{l+s+s1}{\PYZsq{}}\PY{p}{]} 

\PY{c+c1}{\PYZsh{} creating figure}
\PY{n}{fig}\PY{p}{,} \PY{n}{ax} \PY{o}{=} \PY{n}{plt}\PY{o}{.}\PY{n}{subplots}\PY{p}{(}\PY{l+m+mi}{1}\PY{p}{,}\PY{l+m+mi}{2}\PY{p}{,} \PY{n}{figsize}\PY{o}{=}\PY{p}{(}\PY{n}{width}\PY{p}{,}\PY{l+m+mi}{7}\PY{p}{)}\PY{p}{)}
\PY{n}{sns}\PY{o}{.}\PY{n}{set\PYZus{}style}\PY{p}{(}\PY{l+s+s1}{\PYZsq{}}\PY{l+s+s1}{whitegrid}\PY{l+s+s1}{\PYZsq{}}\PY{p}{)}
\PY{n}{title} \PY{o}{=} \PY{n}{fig}\PY{o}{.}\PY{n}{suptitle}\PY{p}{(}\PY{l+s+s1}{\PYZsq{}}\PY{l+s+s1}{Marktanteile der Top 5 Marken}\PY{l+s+s1}{\PYZsq{}}\PY{p}{,} \PY{n}{fontsize}\PY{o}{=} \PY{n}{title\PYZus{}fontsize}\PY{p}{,} \PY{n}{fontweight}\PY{o}{=} \PY{l+s+s1}{\PYZsq{}}\PY{l+s+s1}{bold}\PY{l+s+s1}{\PYZsq{}}\PY{p}{)}
\PY{n}{title}\PY{o}{.}\PY{n}{set\PYZus{}x}\PY{p}{(}\PY{l+m+mf}{0.18}\PY{p}{)}

\PY{c+c1}{\PYZsh{} barplot}
\PY{n}{sns}\PY{o}{.}\PY{n}{barplot}\PY{p}{(}\PY{n}{data}\PY{o}{=}\PY{n}{top\PYZus{}brands}\PY{p}{,} \PY{n}{y}\PY{o}{=}\PY{l+s+s1}{\PYZsq{}}\PY{l+s+s1}{brand}\PY{l+s+s1}{\PYZsq{}}\PY{p}{,} \PY{n}{x}\PY{o}{=}\PY{l+s+s1}{\PYZsq{}}\PY{l+s+s1}{count}\PY{l+s+s1}{\PYZsq{}}\PY{p}{,} \PY{n}{ax}\PY{o}{=} \PY{n}{ax}\PY{p}{[}\PY{l+m+mi}{0}\PY{p}{]}\PY{p}{,} \PY{n}{palette}\PY{o}{=} \PY{n}{custom\PYZus{}palette}\PY{p}{)}

\PY{c+c1}{\PYZsh{} Add count inside each bar}
\PY{k}{for} \PY{n}{i}\PY{p}{,} \PY{n}{v} \PY{o+ow}{in} \PY{n+nb}{enumerate}\PY{p}{(}\PY{n}{top\PYZus{}brands}\PY{p}{[}\PY{l+s+s1}{\PYZsq{}}\PY{l+s+s1}{count}\PY{l+s+s1}{\PYZsq{}}\PY{p}{]}\PY{p}{)}\PY{p}{:}
    \PY{n}{ax}\PY{p}{[}\PY{l+m+mi}{0}\PY{p}{]}\PY{o}{.}\PY{n}{annotate}\PY{p}{(}\PY{n+nb}{str}\PY{p}{(}\PY{n}{v}\PY{p}{)}\PY{p}{,} \PY{n}{xy}\PY{o}{=}\PY{p}{(}\PY{n}{v}\PY{p}{,} \PY{n}{i}\PY{p}{)}\PY{p}{,} \PY{n}{va}\PY{o}{=}\PY{l+s+s1}{\PYZsq{}}\PY{l+s+s1}{center}\PY{l+s+s1}{\PYZsq{}}\PY{p}{,} \PY{n}{ha}\PY{o}{=}\PY{l+s+s1}{\PYZsq{}}\PY{l+s+s1}{right}\PY{l+s+s1}{\PYZsq{}}\PY{p}{,} \PY{n}{fontsize}\PY{o}{=}\PY{l+m+mi}{10}\PY{p}{,} \PY{n}{color}\PY{o}{=}\PY{l+s+s1}{\PYZsq{}}\PY{l+s+s1}{white}\PY{l+s+s1}{\PYZsq{}}\PY{p}{,} 
                \PY{n}{xytext}\PY{o}{=}\PY{p}{(}\PY{o}{\PYZhy{}}\PY{l+m+mi}{5}\PY{p}{,} \PY{l+m+mi}{0}\PY{p}{)}\PY{p}{,} \PY{n}{textcoords}\PY{o}{=}\PY{l+s+s1}{\PYZsq{}}\PY{l+s+s1}{offset points}\PY{l+s+s1}{\PYZsq{}}\PY{p}{,} \PY{n}{fontweight}\PY{o}{=} \PY{l+s+s1}{\PYZsq{}}\PY{l+s+s1}{bold}\PY{l+s+s1}{\PYZsq{}}\PY{p}{)}

\PY{c+c1}{\PYZsh{} Set plot attributes}
\PY{n}{ax}\PY{p}{[}\PY{l+m+mi}{0}\PY{p}{]}\PY{o}{.}\PY{n}{set}\PY{p}{(}\PY{n}{title}\PY{o}{=}\PY{l+s+s1}{\PYZsq{}}\PY{l+s+s1}{Top 5 Marken}\PY{l+s+s1}{\PYZsq{}}\PY{p}{,} \PY{n}{xlabel}\PY{o}{=}\PY{l+s+s1}{\PYZsq{}}\PY{l+s+s1}{Anzahl}\PY{l+s+s1}{\PYZsq{}}\PY{p}{,} \PY{n}{ylabel}\PY{o}{=}\PY{l+s+s1}{\PYZsq{}}\PY{l+s+s1}{Marke}\PY{l+s+s1}{\PYZsq{}}\PY{p}{)}

\PY{c+c1}{\PYZsh{} add rest}
\PY{n}{top\PYZus{}brands} \PY{o}{=} \PY{n}{top\PYZus{}brands}\PY{o}{.}\PY{n}{append}\PY{p}{(}\PY{n}{rest\PYZus{}brands}\PY{p}{)}

\PY{c+c1}{\PYZsh{} Create a donut }
\PY{n}{sizes} \PY{o}{=} \PY{n}{top\PYZus{}brands}\PY{p}{[}\PY{l+s+s1}{\PYZsq{}}\PY{l+s+s1}{count}\PY{l+s+s1}{\PYZsq{}}\PY{p}{]}
\PY{n}{labels} \PY{o}{=} \PY{n}{top\PYZus{}brands}\PY{p}{[}\PY{l+s+s1}{\PYZsq{}}\PY{l+s+s1}{brand}\PY{l+s+s1}{\PYZsq{}}\PY{p}{]}
\PY{n}{colors} \PY{o}{=} \PY{n}{custom\PYZus{}palette}\PY{p}{[}\PY{p}{:}\PY{n+nb}{len}\PY{p}{(}\PY{n}{sizes}\PY{p}{)}\PY{p}{]}

\PY{c+c1}{\PYZsh{} white circle}
\PY{n}{wedgeprops} \PY{o}{=} \PY{p}{\PYZob{}}\PY{l+s+s2}{\PYZdq{}}\PY{l+s+s2}{edgecolor}\PY{l+s+s2}{\PYZdq{}}\PY{p}{:} \PY{l+s+s2}{\PYZdq{}}\PY{l+s+s2}{w}\PY{l+s+s2}{\PYZdq{}}\PY{p}{,} \PY{l+s+s1}{\PYZsq{}}\PY{l+s+s1}{linewidth}\PY{l+s+s1}{\PYZsq{}}\PY{p}{:} \PY{l+m+mi}{1}\PY{p}{,} \PY{l+s+s1}{\PYZsq{}}\PY{l+s+s1}{width}\PY{l+s+s1}{\PYZsq{}}\PY{p}{:} \PY{l+m+mf}{0.35}\PY{p}{\PYZcb{}}
\PY{c+c1}{\PYZsh{} set pie}
\PY{n}{\PYZus{}}\PY{p}{,} \PY{n}{lab}\PY{p}{,} \PY{n}{pct} \PY{o}{=} \PY{n}{ax}\PY{p}{[}\PY{l+m+mi}{1}\PY{p}{]}\PY{o}{.}\PY{n}{pie}\PY{p}{(}\PY{n}{sizes}\PY{p}{,} \PY{n}{labels}\PY{o}{=}\PY{n}{labels}\PY{p}{,} \PY{n}{colors}\PY{o}{=}\PY{n}{colors}\PY{p}{,} \PY{n}{startangle}\PY{o}{=}\PY{l+m+mi}{90}\PY{p}{,} 
                    \PY{n}{wedgeprops}\PY{o}{=}\PY{n}{wedgeprops}\PY{p}{,} \PY{n}{autopct}\PY{o}{=}\PY{l+s+s1}{\PYZsq{}}\PY{l+s+si}{\PYZpc{}1.1f}\PY{l+s+si}{\PYZpc{}\PYZpc{}}\PY{l+s+s1}{\PYZsq{}}\PY{p}{,} \PY{n}{pctdistance}\PY{o}{=}\PY{l+m+mf}{0.82}\PY{p}{)}

\PY{c+c1}{\PYZsh{} set the color of the labels to red}
\PY{n}{plt}\PY{o}{.}\PY{n}{setp}\PY{p}{(}\PY{n}{lab}\PY{p}{,} \PY{n}{color}\PY{o}{=}\PY{l+s+s1}{\PYZsq{}}\PY{l+s+s1}{black}\PY{l+s+s1}{\PYZsq{}}\PY{p}{,} \PY{n}{fontsize}\PY{o}{=} \PY{l+m+mi}{12}\PY{p}{)}
\PY{n}{plt}\PY{o}{.}\PY{n}{setp}\PY{p}{(}\PY{n}{pct}\PY{p}{,} \PY{n}{color}\PY{o}{=}\PY{l+s+s1}{\PYZsq{}}\PY{l+s+s1}{white}\PY{l+s+s1}{\PYZsq{}}\PY{p}{,} \PY{n}{fontsize}\PY{o}{=} \PY{l+m+mi}{10}\PY{p}{,} \PY{n}{fontweight}\PY{o}{=} \PY{l+s+s1}{\PYZsq{}}\PY{l+s+s1}{bold}\PY{l+s+s1}{\PYZsq{}}\PY{p}{)}

\PY{c+c1}{\PYZsh{} Set plot attributes}
\PY{n}{ax}\PY{p}{[}\PY{l+m+mi}{1}\PY{p}{]}\PY{o}{.}\PY{n}{set}\PY{p}{(}\PY{n}{title}\PY{o}{=}\PY{l+s+s1}{\PYZsq{}}\PY{l+s+s1}{Prozentuale Verteilung}\PY{l+s+s1}{\PYZsq{}}\PY{p}{,} \PY{n}{xlabel}\PY{o}{=}\PY{l+s+s1}{\PYZsq{}}\PY{l+s+s1}{\PYZsq{}}\PY{p}{,} \PY{n}{ylabel}\PY{o}{=}\PY{l+s+s1}{\PYZsq{}}\PY{l+s+s1}{\PYZsq{}}\PY{p}{)}

\PY{c+c1}{\PYZsh{} Show the plot}
\PY{n}{plt}\PY{o}{.}\PY{n}{show}\PY{p}{(}\PY{p}{)}
\end{Verbatim}
\end{tcolorbox}

    \begin{center}
\hspace{-2cm}
    \adjustimage{max size={0.9\linewidth}{0.9\paperheight}}{output_22_1.png}
    \end{center}

Das Balkendiagramm ermöglicht es uns, schnell die Anzahl
der Tankstellen für jede Marke zu vergleichen und festzustellen, welche
Marken den größten Marktanteil haben. Die Hinzufügung der Anzahl-Werte
innerhalb jedes Balkens liefert detailliertere Informationen und
ermöglicht es uns, die genaue Anzahl der Tankstellen für
jede~Marke~zu~sehen. Der Donut-Plot auf der rechten Seite gibt
Aufschluss auf die prozentualen Marktanteile der Top 5 und ergänzt den
Vergleich mit den restlichen Tankstellen.


    { \hspace*{\fill} \\}
    \hypertarget{geografische-visualisierung-der-top-5-tankstellenmarken-in-deutschland}{%
\subsection{Geografische Visualisierung der Top 5 Tankstellenmarken in
Deutschland}\label{geografische-visualisierung-der-top-5-tankstellenmarken-in-deutschland}}

    Nachdem wir nun wissen, welche Tankstellenmarken in Deutschland am
stärksten vertreten sind, ist die geografische Verteilung
dieser~von~Interesse. Die folgenden Plots stellen die Tankstellen der
größten 5 Marken sowohl als Gesamtübersicht, als auch aufgeschlüsselt
pro Marke, dar.

    \begin{tcolorbox}[breakable, size=fbox, boxrule=1pt, pad at break*=1mm,colback=cellbackground, colframe=cellborder]
\prompt{In}{incolor}{5}{\boxspacing}
\begin{Verbatim}[commandchars=\\\{\}]
\PY{c+c1}{\PYZsh{} Select only the top 5 brands and don\PYZsq{}t sort by count}
\PY{n}{top\PYZus{}brands} \PY{o}{=} \PY{n}{stations}\PY{p}{[}\PY{l+s+s1}{\PYZsq{}}\PY{l+s+s1}{brand}\PY{l+s+s1}{\PYZsq{}}\PY{p}{]}\PY{o}{.}\PY{n}{value\PYZus{}counts}\PY{p}{(}\PY{n}{sort}\PY{o}{=}\PY{k+kc}{False}\PY{p}{)}\PY{o}{.}\PY{n}{nlargest}\PY{p}{(}\PY{l+m+mi}{5}\PY{p}{)}\PY{o}{.}\PY{n}{index}\PY{o}{.}\PY{n}{tolist}\PY{p}{(}\PY{p}{)}
\PY{n}{df\PYZus{}map} \PY{o}{=} \PY{n}{stations}\PY{p}{[}\PY{n}{stations}\PY{p}{[}\PY{l+s+s1}{\PYZsq{}}\PY{l+s+s1}{brand}\PY{l+s+s1}{\PYZsq{}}\PY{p}{]}\PY{o}{.}\PY{n}{isin}\PY{p}{(}\PY{n}{top\PYZus{}brands}\PY{p}{)}\PY{p}{]}

\PY{n}{fig}\PY{p}{,} \PY{n}{axes} \PY{o}{=} \PY{n}{plt}\PY{o}{.}\PY{n}{subplots}\PY{p}{(}\PY{l+m+mi}{2}\PY{p}{,}\PY{l+m+mi}{3}\PY{p}{,} \PY{n}{figsize}\PY{o}{=}\PY{p}{(}\PY{n}{width}\PY{p}{,}\PY{l+m+mi}{11}\PY{p}{)}\PY{p}{)}
\PY{n}{sns}\PY{o}{.}\PY{n}{set\PYZus{}style}\PY{p}{(}\PY{l+s+s1}{\PYZsq{}}\PY{l+s+s1}{whitegrid}\PY{l+s+s1}{\PYZsq{}}\PY{p}{)}

\PY{c+c1}{\PYZsh{} add a title to the map}
\PY{n}{title} \PY{o}{=} \PY{n}{fig}\PY{o}{.}\PY{n}{suptitle}\PY{p}{(}\PY{l+s+s1}{\PYZsq{}}\PY{l+s+s1}{Verteilung der Top 5 Tankstellenmarken in Deutschland}\PY{l+s+s1}{\PYZsq{}}\PY{p}{,} \PY{n}{fontsize}\PY{o}{=} \PY{n}{title\PYZus{}fontsize}\PY{p}{,} \PY{n}{fontweight}\PY{o}{=} \PY{l+s+s1}{\PYZsq{}}\PY{l+s+s1}{bold}\PY{l+s+s1}{\PYZsq{}}\PY{p}{)}

\PY{n}{shapefile\PYZus{}data} \PY{o}{=} \PY{n}{shapefile}\PY{o}{.}\PY{n}{Reader}\PY{p}{(}\PY{l+s+s1}{\PYZsq{}}\PY{l+s+s1}{DEU\PYZus{}adm0.shp}\PY{l+s+s1}{\PYZsq{}}\PY{p}{)}

\PY{c+c1}{\PYZsh{} create legend labels and colors}
\PY{n}{stations\PYZus{}loop} \PY{o}{=} \PY{p}{[}\PY{l+s+s1}{\PYZsq{}}\PY{l+s+s1}{Top 5}\PY{l+s+s1}{\PYZsq{}}\PY{p}{]} \PY{o}{+} \PY{n}{top\PYZus{}brands}
\PY{n}{legend\PYZus{}labels} \PY{o}{=} \PY{n}{top\PYZus{}brands}
\PY{n}{legend\PYZus{}colors} \PY{o}{=} \PY{p}{[}\PY{n}{custom\PYZus{}palette}\PY{p}{[}\PY{n}{top\PYZus{}brands}\PY{o}{.}\PY{n}{index}\PY{p}{(}\PY{n}{brand}\PY{p}{)}\PY{p}{]} \PY{k}{for} \PY{n}{brand} \PY{o+ow}{in} \PY{n}{top\PYZus{}brands}\PY{p}{]}

\PY{c+c1}{\PYZsh{} adding dots based on latitudes and longitudes}
\PY{k}{for} \PY{n}{ax}\PY{p}{,} \PY{n}{string} \PY{o+ow}{in} \PY{n+nb}{zip}\PY{p}{(}\PY{n}{axes}\PY{o}{.}\PY{n}{flatten}\PY{p}{(}\PY{p}{)}\PY{p}{,}\PY{n}{stations\PYZus{}loop}\PY{p}{)}\PY{p}{:}
    \PY{c+c1}{\PYZsh{} drawn shape file}
    \PY{n}{poly}\PY{o}{=} \PY{n}{shapefile\PYZus{}data}\PY{o}{.}\PY{n}{shape}\PY{p}{(}\PY{p}{)}
    \PY{n}{ax}\PY{o}{.}\PY{n}{add\PYZus{}patch}\PY{p}{(}\PY{n}{PolygonPatch}\PY{p}{(}\PY{n}{poly}\PY{p}{,} \PY{n}{fc}\PY{o}{=}\PY{l+s+s1}{\PYZsq{}}\PY{l+s+s1}{\PYZsh{}ffffff}\PY{l+s+s1}{\PYZsq{}}\PY{p}{,} \PY{n}{ec}\PY{o}{=}\PY{l+s+s1}{\PYZsq{}}\PY{l+s+s1}{\PYZsh{}000000}\PY{l+s+s1}{\PYZsq{}}\PY{p}{,} \PY{n}{alpha}\PY{o}{=}\PY{l+m+mf}{0.5}\PY{p}{,} \PY{n}{zorder}\PY{o}{=}\PY{l+m+mi}{2} \PY{p}{)}\PY{p}{)}

    \PY{c+c1}{\PYZsh{} differ betwenn all and seperate}
    \PY{k}{if} \PY{n}{string} \PY{o}{==} \PY{l+s+s1}{\PYZsq{}}\PY{l+s+s1}{Top 5}\PY{l+s+s1}{\PYZsq{}}\PY{p}{:}   
        \PY{c+c1}{\PYZsh{} print dots}
        \PY{k}{for} \PY{n}{t}\PY{p}{,} \PY{n}{row} \PY{o+ow}{in} \PY{n}{df\PYZus{}map}\PY{o}{.}\PY{n}{iterrows}\PY{p}{(}\PY{p}{)}\PY{p}{:}
            \PY{n}{ax}\PY{o}{.}\PY{n}{plot}\PY{p}{(}\PY{n}{row}\PY{p}{[}\PY{l+s+s1}{\PYZsq{}}\PY{l+s+s1}{longitude}\PY{l+s+s1}{\PYZsq{}}\PY{p}{]}\PY{p}{,} \PY{n}{row}\PY{p}{[}\PY{l+s+s1}{\PYZsq{}}\PY{l+s+s1}{latitude}\PY{l+s+s1}{\PYZsq{}}\PY{p}{]}\PY{p}{,} \PY{n}{marker}\PY{o}{=}\PY{l+s+s1}{\PYZsq{}}\PY{l+s+s1}{o}\PY{l+s+s1}{\PYZsq{}}\PY{p}{,} \PY{n}{markersize}\PY{o}{=}\PY{l+m+mi}{2}\PY{p}{,} 
                    \PY{n}{color}\PY{o}{=}\PY{n}{custom\PYZus{}palette}\PY{p}{[}\PY{n}{top\PYZus{}brands}\PY{o}{.}\PY{n}{index}\PY{p}{(}\PY{n}{row}\PY{p}{[}\PY{l+s+s1}{\PYZsq{}}\PY{l+s+s1}{brand}\PY{l+s+s1}{\PYZsq{}}\PY{p}{]}\PY{p}{)}\PY{p}{]}\PY{p}{)}
            
        \PY{c+c1}{\PYZsh{} add legend without color}
        \PY{n}{handles\PYZus{}all} \PY{o}{=} \PY{p}{[}\PY{n}{plt}\PY{o}{.}\PY{n}{Line2D}\PY{p}{(}\PY{p}{[}\PY{p}{]}\PY{p}{,} \PY{p}{[}\PY{p}{]}\PY{p}{,} \PY{n}{linestyle}\PY{o}{=}\PY{l+s+s1}{\PYZsq{}}\PY{l+s+s1}{\PYZsq{}}\PY{p}{,} \PY{n}{label}\PY{o}{=}\PY{n}{string}\PY{p}{,} \PY{n}{marker}\PY{o}{=} \PY{l+s+s1}{\PYZsq{}}\PY{l+s+s1}{None}\PY{l+s+s1}{\PYZsq{}}\PY{p}{)}\PY{p}{]}
        \PY{n}{ax}\PY{o}{.}\PY{n}{legend}\PY{p}{(}\PY{n}{handles}\PY{o}{=}\PY{n}{handles\PYZus{}all}\PY{p}{,} \PY{n}{loc}\PY{o}{=}\PY{l+s+s1}{\PYZsq{}}\PY{l+s+s1}{upper center}\PY{l+s+s1}{\PYZsq{}}\PY{p}{,} \PY{n}{ncol}\PY{o}{=}\PY{n+nb}{len}\PY{p}{(}\PY{n}{legend\PYZus{}labels}\PY{p}{)}\PY{p}{,} \PY{n}{labelcolor}\PY{o}{=}\PY{l+s+s1}{\PYZsq{}}\PY{l+s+s1}{black}\PY{l+s+s1}{\PYZsq{}}\PY{p}{,} \PY{n}{frameon}\PY{o}{=}\PY{k+kc}{False}\PY{p}{)}

    \PY{k}{else}\PY{p}{:}
        \PY{c+c1}{\PYZsh{} print dots}
        \PY{k}{for} \PY{n}{t}\PY{p}{,} \PY{n}{row} \PY{o+ow}{in} \PY{n}{df\PYZus{}map}\PY{p}{[}\PY{n}{df\PYZus{}map}\PY{o}{.}\PY{n}{brand} \PY{o}{==} \PY{n}{string}\PY{p}{]}\PY{o}{.}\PY{n}{iterrows}\PY{p}{(}\PY{p}{)}\PY{p}{:}
            \PY{n}{ax}\PY{o}{.}\PY{n}{plot}\PY{p}{(}\PY{n}{row}\PY{p}{[}\PY{l+s+s1}{\PYZsq{}}\PY{l+s+s1}{longitude}\PY{l+s+s1}{\PYZsq{}}\PY{p}{]}\PY{p}{,} \PY{n}{row}\PY{p}{[}\PY{l+s+s1}{\PYZsq{}}\PY{l+s+s1}{latitude}\PY{l+s+s1}{\PYZsq{}}\PY{p}{]}\PY{p}{,} \PY{n}{marker}\PY{o}{=}\PY{l+s+s1}{\PYZsq{}}\PY{l+s+s1}{o}\PY{l+s+s1}{\PYZsq{}}\PY{p}{,} \PY{n}{markersize}\PY{o}{=}\PY{l+m+mi}{2}\PY{p}{,} 
                    \PY{n}{color}\PY{o}{=}\PY{n}{custom\PYZus{}palette}\PY{p}{[}\PY{n}{top\PYZus{}brands}\PY{o}{.}\PY{n}{index}\PY{p}{(}\PY{n}{row}\PY{p}{[}\PY{l+s+s1}{\PYZsq{}}\PY{l+s+s1}{brand}\PY{l+s+s1}{\PYZsq{}}\PY{p}{]}\PY{p}{)}\PY{p}{]}\PY{p}{)}
        
        \PY{c+c1}{\PYZsh{} add legend to the map}
        \PY{n}{handles} \PY{o}{=} \PY{p}{[}\PY{n}{plt}\PY{o}{.}\PY{n}{Line2D}\PY{p}{(}\PY{p}{[}\PY{p}{]}\PY{p}{,} \PY{p}{[}\PY{p}{]}\PY{p}{,} \PY{n}{linestyle}\PY{o}{=}\PY{l+s+s1}{\PYZsq{}}\PY{l+s+s1}{\PYZhy{}}\PY{l+s+s1}{\PYZsq{}}\PY{p}{,} \PY{n}{label}\PY{o}{=}\PY{n}{string}\PY{p}{,} \PY{n}{color}\PY{o}{=}\PY{n}{custom\PYZus{}palette}\PY{p}{[}\PY{n}{top\PYZus{}brands}\PY{o}{.}\PY{n}{index}\PY{p}{(}\PY{n}{string}\PY{p}{)}\PY{p}{]}\PY{p}{)}\PY{p}{]}
        \PY{n}{ax}\PY{o}{.}\PY{n}{legend}\PY{p}{(}\PY{n}{handles}\PY{o}{=}\PY{n}{handles}\PY{p}{,} \PY{n}{loc}\PY{o}{=}\PY{l+s+s1}{\PYZsq{}}\PY{l+s+s1}{upper center}\PY{l+s+s1}{\PYZsq{}}\PY{p}{,} \PY{n}{ncol}\PY{o}{=}\PY{n+nb}{len}\PY{p}{(}\PY{n}{legend\PYZus{}labels}\PY{p}{)}\PY{p}{,} \PY{n}{labelcolor}\PY{o}{=}\PY{n}{custom\PYZus{}palette}\PY{p}{[}\PY{n}{top\PYZus{}brands}\PY{o}{.}\PY{n}{index}\PY{p}{(}\PY{n}{string}\PY{p}{)}\PY{p}{]}\PY{p}{,} \PY{n}{frameon}\PY{o}{=}\PY{k+kc}{False}\PY{p}{)}
            
    \PY{c+c1}{\PYZsh{} Remove stuff around}
    \PY{n}{ax}\PY{o}{.}\PY{n}{grid}\PY{p}{(}\PY{k+kc}{False}\PY{p}{)}
    \PY{n}{ax}\PY{o}{.}\PY{n}{set\PYZus{}xticks}\PY{p}{(}\PY{p}{[}\PY{p}{]}\PY{p}{)}
    \PY{n}{ax}\PY{o}{.}\PY{n}{set\PYZus{}yticks}\PY{p}{(}\PY{p}{[}\PY{p}{]}\PY{p}{)}

\PY{c+c1}{\PYZsh{} Adjust spacing}
\PY{n}{fig}\PY{o}{.}\PY{n}{subplots\PYZus{}adjust}\PY{p}{(}\PY{n}{top}\PY{o}{=}\PY{l+m+mf}{0.94}\PY{p}{,} \PY{n}{hspace}\PY{o}{=}\PY{l+m+mi}{0} \PY{p}{,} \PY{n}{wspace}\PY{o}{=}\PY{l+m+mi}{0}\PY{p}{)}
\PY{n}{title}\PY{o}{.}\PY{n}{set\PYZus{}y}\PY{p}{(}\PY{l+m+mf}{0.97}\PY{p}{)}
\PY{n}{title}\PY{o}{.}\PY{n}{set\PYZus{}x}\PY{p}{(}\PY{l+m+mf}{0.388}\PY{p}{)}

\PY{n}{plt}\PY{o}{.}\PY{n}{show}\PY{p}{(}\PY{p}{)}
\end{Verbatim}
\end{tcolorbox}

    Deutschland verfügt über ein großflächiges Tankstellennetz, das sich
jedoch ungleichmäßig auf verschiedene Regionen verteilt. Insbesondere im
Ruhrgebiet, Berlin und Frankfurt sowie weiteren dicht besiedelten
Gebieten gibt es zahlreiche Tankstellen. Im Vergleich dazu ist die
Anzahl der Tankstellen in weniger besiedelten Regionen, wie
Mecklenburg-Vorpommern vergleichsweise gering. Es gibt auch Unterschiede
zwischen Ost und West, da im Osten Deutschlands weniger Tankstellen
vorhanden sind und nicht alle Top-5-Tankstellenmarken, wie zum Beispiel
AVIA, dort vertreten sind. Die Geomap zeigt somit die Verteilung der
Top-5-Tankstellenmarken in Deutschland auf und hebt die heutzutage in
Teilen immer noch bestehenden infrastrukturellen Unterschiede zwischen
Ost und West hervor.

\begin{center}
\hspace{-2cm}
    \adjustimage{max size={0.9\linewidth}{0.9\paperheight}}{output_25_0.png}
    \end{center}    




    \hypertarget{aktueller-spritpreisvergleich-der-marken}{%
\subsection{Aktueller Spritpreisvergleich der
Marken}\label{aktueller-spritpreisvergleich-der-marken}}

    Wir hatten die Vermutung, dass Kraftstoff-Preise in den neuen Länder
geringer sein könnten, als in den alten Ländern wie es öfter der Falls
ist z.B. bei Grundstücken. Um dies zu überprüfen, erstellen wir ein
Balken-Diagramm mit den durchschnittlichen Preisen der Kraftstoffe je
Bundesland. Dabei bezieht sich der Durchschnitt auf die Preise im März
des Jahres 2023. Da der ursprüngliche Datensatz keine Bundesländer
enthält, wird ein externer Datensatz benötigt, der jeder Postleitzahl
ein Bundesland zuordnet. Die Postleitzahl wird benötigt, um sie als
Schlüssel zum Vereinen der beiden Datensetze zu verwenden. Man hätte
auch die Stadt als Schlüssel benutzen können, allerdings besteht die
Möglichkeit, dass verschiedene Städte den gleichen Namen tragen. Daher
haben wir uns für die eindeutigen Postleitzahlen entschieden. Jedoch
könnte nicht nur der Vergleich zwischen Ost- und West-Ländern
interessant sein, sondern auch der Vergleich zwischen allen 16
Bundesländern, weswegen wir die Preise nur bis zur Bundesländer-Ebene
aggregieren.

    \begin{tcolorbox}[breakable, size=fbox, boxrule=1pt, pad at break*=1mm,colback=cellbackground, colframe=cellborder]
\prompt{In}{incolor}{6}{\boxspacing}
\begin{Verbatim}[commandchars=\\\{\}]
\PY{c+c1}{\PYZsh{} mean per station}
\PY{n}{month\PYZus{}means} \PY{o}{=} \PY{n}{april\PYZus{}2023}\PY{o}{.}\PY{n}{groupby}\PY{p}{(}\PY{l+s+s1}{\PYZsq{}}\PY{l+s+s1}{station\PYZus{}uuid}\PY{l+s+s1}{\PYZsq{}}\PY{p}{)}\PY{o}{.}\PY{n}{mean}\PY{p}{(}\PY{p}{)}\PY{o}{.}\PY{n}{reset\PYZus{}index}\PY{p}{(}\PY{p}{)}

\PY{c+c1}{\PYZsh{} adding postcode to stations}
\PY{n}{month\PYZus{}means} \PY{o}{=} \PY{n}{month\PYZus{}means}\PY{o}{.}\PY{n}{merge}\PY{p}{(}\PY{n}{stations}\PY{p}{[}\PY{p}{[}\PY{l+s+s1}{\PYZsq{}}\PY{l+s+s1}{station\PYZus{}uuid}\PY{l+s+s1}{\PYZsq{}}\PY{p}{,} \PY{l+s+s1}{\PYZsq{}}\PY{l+s+s1}{post\PYZus{}code}\PY{l+s+s1}{\PYZsq{}}\PY{p}{]}\PY{p}{]}\PY{p}{,} \PY{n}{on}\PY{o}{=}\PY{l+s+s1}{\PYZsq{}}\PY{l+s+s1}{station\PYZus{}uuid}\PY{l+s+s1}{\PYZsq{}}\PY{p}{)}

\PY{c+c1}{\PYZsh{} mean by postcode}
\PY{n}{month\PYZus{}postcode\PYZus{}means} \PY{o}{=} \PY{n}{month\PYZus{}means}\PY{o}{.}\PY{n}{astype}\PY{p}{(}\PY{p}{\PYZob{}}\PY{l+s+s1}{\PYZsq{}}\PY{l+s+s1}{post\PYZus{}code}\PY{l+s+s1}{\PYZsq{}}\PY{p}{:} \PY{l+s+s1}{\PYZsq{}}\PY{l+s+s1}{int}\PY{l+s+s1}{\PYZsq{}}\PY{p}{\PYZcb{}}\PY{p}{)}\PY{o}{.}\PY{n}{groupby}\PY{p}{(}\PY{l+s+s1}{\PYZsq{}}\PY{l+s+s1}{post\PYZus{}code}\PY{l+s+s1}{\PYZsq{}}\PY{p}{)}\PY{o}{.}\PY{n}{mean}\PY{p}{(}\PY{p}{)}\PY{o}{.}\PY{n}{reset\PYZus{}index}\PY{p}{(}\PY{p}{)}

\PY{c+c1}{\PYZsh{} merging postcode to Bundesländer + calculating mean of it }
\PY{n}{plz\PYZus{}bundesland} \PY{o}{=} \PY{n}{pd}\PY{o}{.}\PY{n}{read\PYZus{}csv}\PY{p}{(}\PY{l+s+s1}{\PYZsq{}}\PY{l+s+s1}{plz\PYZhy{}buland\PYZhy{}kreis.csv}\PY{l+s+s1}{\PYZsq{}}\PY{p}{,} \PY{n}{delimiter}\PY{o}{=} \PY{l+s+s1}{\PYZsq{}}\PY{l+s+s1}{\PYZbs{}}\PY{l+s+s1}{;}\PY{l+s+s1}{\PYZsq{}}\PY{p}{)}
\PY{n}{plz\PYZus{}bundesland} \PY{o}{=} \PY{n}{plz\PYZus{}bundesland}\PY{p}{[}\PY{p}{[}\PY{l+s+s1}{\PYZsq{}}\PY{l+s+s1}{plz}\PY{l+s+s1}{\PYZsq{}}\PY{p}{,} \PY{l+s+s1}{\PYZsq{}}\PY{l+s+s1}{bundesland}\PY{l+s+s1}{\PYZsq{}}\PY{p}{]}\PY{p}{]}\PY{o}{.}\PY{n}{drop\PYZus{}duplicates}\PY{p}{(}\PY{p}{)}\PY{o}{.}\PY{n}{rename}\PY{p}{(}\PY{n}{columns}\PY{o}{=}\PY{p}{\PYZob{}}\PY{l+s+s1}{\PYZsq{}}\PY{l+s+s1}{plz}\PY{l+s+s1}{\PYZsq{}}\PY{p}{:} \PY{l+s+s1}{\PYZsq{}}\PY{l+s+s1}{post\PYZus{}code}\PY{l+s+s1}{\PYZsq{}}\PY{p}{\PYZcb{}}\PY{p}{)}
\PY{n}{bundland\PYZus{}preise} \PY{o}{=} \PY{n}{month\PYZus{}postcode\PYZus{}means}\PY{o}{.}\PY{n}{merge}\PY{p}{(}\PY{n}{plz\PYZus{}bundesland}\PY{p}{,} \PY{n}{on}\PY{o}{=}\PY{l+s+s1}{\PYZsq{}}\PY{l+s+s1}{post\PYZus{}code}\PY{l+s+s1}{\PYZsq{}}\PY{p}{)}\PY{o}{.}\PY{n}{groupby}\PY{p}{(}\PY{l+s+s1}{\PYZsq{}}\PY{l+s+s1}{bundesland}\PY{l+s+s1}{\PYZsq{}}\PY{p}{)}\PY{o}{.}\PY{n}{mean}\PY{p}{(}\PY{p}{)}\PY{o}{.}\PY{n}{reset\PYZus{}index}\PY{p}{(}\PY{p}{)}

\PY{c+c1}{\PYZsh{} creating a figure to have three barplots next to each other + title}
\PY{n}{fig}\PY{p}{,} \PY{n}{ax} \PY{o}{=} \PY{n}{plt}\PY{o}{.}\PY{n}{subplots}\PY{p}{(}\PY{l+m+mi}{1}\PY{p}{,} \PY{l+m+mi}{3}\PY{p}{,} \PY{n}{figsize}\PY{o}{=}\PY{p}{(}\PY{n}{width}\PY{p}{,} \PY{l+m+mi}{11}\PY{p}{)}\PY{p}{)}
\PY{n}{sns}\PY{o}{.}\PY{n}{set\PYZus{}style}\PY{p}{(}\PY{l+s+s1}{\PYZsq{}}\PY{l+s+s1}{whitegrid}\PY{l+s+s1}{\PYZsq{}}\PY{p}{)}
\PY{n}{title} \PY{o}{=} \PY{n}{fig}\PY{o}{.}\PY{n}{suptitle}\PY{p}{(}\PY{l+s+s1}{\PYZsq{}}\PY{l+s+s1}{Ost\PYZhy{}West Vergleich der Treibstoffarten (Durchschnitt April)}\PY{l+s+s1}{\PYZsq{}}\PY{p}{,} \PY{n}{fontsize}\PY{o}{=} \PY{n}{title\PYZus{}fontsize}\PY{p}{,} \PY{n}{fontweight}\PY{o}{=} \PY{l+s+s1}{\PYZsq{}}\PY{l+s+s1}{bold}\PY{l+s+s1}{\PYZsq{}}\PY{p}{)}
\PY{n}{title}\PY{o}{.}\PY{n}{set\PYZus{}x}\PY{p}{(}\PY{l+m+mf}{0.25}\PY{p}{)}

\PY{c+c1}{\PYZsh{} for iterating }
\PY{n}{fuelTypes} \PY{o}{=} \PY{p}{[}\PY{l+s+s1}{\PYZsq{}}\PY{l+s+s1}{diesel}\PY{l+s+s1}{\PYZsq{}}\PY{p}{,} \PY{l+s+s1}{\PYZsq{}}\PY{l+s+s1}{e5}\PY{l+s+s1}{\PYZsq{}}\PY{p}{,} \PY{l+s+s1}{\PYZsq{}}\PY{l+s+s1}{e10}\PY{l+s+s1}{\PYZsq{}}\PY{p}{]} 

\PY{c+c1}{\PYZsh{} color codes for fuel types }
\PY{n}{colors} \PY{o}{=} \PY{p}{[}\PY{l+s+s1}{\PYZsq{}}\PY{l+s+s1}{\PYZsh{}000000}\PY{l+s+s1}{\PYZsq{}}\PY{p}{,} \PY{l+s+s1}{\PYZsq{}}\PY{l+s+s1}{\PYZsh{}1670B9}\PY{l+s+s1}{\PYZsq{}}\PY{p}{,} \PY{l+s+s1}{\PYZsq{}}\PY{l+s+s1}{\PYZsh{}46b340}\PY{l+s+s1}{\PYZsq{}}\PY{p}{]} 

\PY{c+c1}{\PYZsh{} adding transperancy column }
\PY{n}{neue\PYZus{}laender} \PY{o}{=} \PY{p}{[}\PY{l+s+s2}{\PYZdq{}}\PY{l+s+s2}{Brandenburg}\PY{l+s+s2}{\PYZdq{}}\PY{p}{,} \PY{l+s+s2}{\PYZdq{}}\PY{l+s+s2}{Mecklenburg\PYZhy{}Vorpommern}\PY{l+s+s2}{\PYZdq{}}\PY{p}{,} \PY{l+s+s2}{\PYZdq{}}\PY{l+s+s2}{Sachsen}\PY{l+s+s2}{\PYZdq{}}\PY{p}{,} \PY{l+s+s2}{\PYZdq{}}\PY{l+s+s2}{Sachsen\PYZhy{}Anhalt}\PY{l+s+s2}{\PYZdq{}}\PY{p}{,} \PY{l+s+s2}{\PYZdq{}}\PY{l+s+s2}{Thüringen}\PY{l+s+s2}{\PYZdq{}}\PY{p}{]}
\PY{n}{bundland\PYZus{}preise}\PY{p}{[}\PY{l+s+s1}{\PYZsq{}}\PY{l+s+s1}{alpha}\PY{l+s+s1}{\PYZsq{}}\PY{p}{]} \PY{o}{=} \PY{n}{bundland\PYZus{}preise}\PY{p}{[}\PY{l+s+s1}{\PYZsq{}}\PY{l+s+s1}{bundesland}\PY{l+s+s1}{\PYZsq{}}\PY{p}{]}\PY{o}{.}\PY{n}{apply}\PY{p}{(}\PY{k}{lambda} \PY{n}{x}\PY{p}{:} \PY{l+m+mi}{1} \PY{k}{if} \PY{n}{x} \PY{o+ow}{in} \PY{n}{neue\PYZus{}laender} \PY{k}{else} \PY{l+m+mf}{0.6}\PY{p}{)}

\PY{c+c1}{\PYZsh{} plotting a barplot per fuelType }
\PY{k}{for} \PY{n}{i}\PY{p}{,} \PY{n}{fuelType} \PY{o+ow}{in} \PY{n+nb}{enumerate}\PY{p}{(}\PY{n}{fuelTypes}\PY{p}{)}\PY{p}{:}
    \PY{n}{bundland\PYZus{}preise\PYZus{}sorted} \PY{o}{=} \PY{n}{bundland\PYZus{}preise}\PY{o}{.}\PY{n}{sort\PYZus{}values}\PY{p}{(}\PY{n}{by}\PY{o}{=}\PY{n}{fuelType}\PY{p}{,} \PY{n}{ascending}\PY{o}{=}\PY{k+kc}{False}\PY{p}{)}
    \PY{n}{sns}\PY{o}{.}\PY{n}{barplot}\PY{p}{(}\PY{n}{data}\PY{o}{=}\PY{n}{bundland\PYZus{}preise\PYZus{}sorted}\PY{p}{,} \PY{n}{y}\PY{o}{=}\PY{l+s+s1}{\PYZsq{}}\PY{l+s+s1}{bundesland}\PY{l+s+s1}{\PYZsq{}}\PY{p}{,} \PY{n}{x}\PY{o}{=}\PY{n}{fuelType}\PY{p}{,} \PY{n}{color}\PY{o}{=}\PY{n}{colors}\PY{p}{[}\PY{n}{i}\PY{p}{]}\PY{p}{,} \PY{n}{ax}\PY{o}{=}\PY{n}{ax}\PY{p}{[}\PY{n}{i}\PY{p}{]}\PY{p}{,}
                \PY{n}{order}\PY{o}{=}\PY{n}{bundland\PYZus{}preise\PYZus{}sorted}\PY{p}{[}\PY{l+s+s1}{\PYZsq{}}\PY{l+s+s1}{bundesland}\PY{l+s+s1}{\PYZsq{}}\PY{p}{]}\PY{p}{)}
    
    \PY{c+c1}{\PYZsh{} correct the x\PYZhy{}ax}
    \PY{n}{ax}\PY{p}{[}\PY{n}{i}\PY{p}{]}\PY{o}{.}\PY{n}{set\PYZus{}xlim}\PY{p}{(}\PY{l+m+mf}{1.55}\PY{p}{,} \PY{l+m+mf}{1.89}\PY{p}{)}

    \PY{c+c1}{\PYZsh{} modifying alpha based on alpha column and text in the bars with annotate }
    \PY{k}{for} \PY{n}{bar}\PY{p}{,} \PY{n}{alpha} \PY{o+ow}{in} \PY{n+nb}{zip}\PY{p}{(}\PY{n}{ax}\PY{p}{[}\PY{n}{i}\PY{p}{]}\PY{o}{.}\PY{n}{patches}\PY{p}{,} \PY{n}{bundland\PYZus{}preise\PYZus{}sorted}\PY{p}{[}\PY{l+s+s1}{\PYZsq{}}\PY{l+s+s1}{alpha}\PY{l+s+s1}{\PYZsq{}}\PY{p}{]}\PY{p}{)}\PY{p}{:}
        \PY{n}{bar}\PY{o}{.}\PY{n}{set\PYZus{}alpha}\PY{p}{(}\PY{n}{alpha}\PY{p}{)}
        \PY{n}{ax}\PY{p}{[}\PY{n}{i}\PY{p}{]}\PY{o}{.}\PY{n}{annotate}\PY{p}{(}\PY{l+s+s2}{\PYZdq{}}\PY{l+s+si}{\PYZpc{}.3f}\PY{l+s+s2}{\PYZdq{}} \PY{o}{\PYZpc{}} \PY{n}{bar}\PY{o}{.}\PY{n}{get\PYZus{}width}\PY{p}{(}\PY{p}{)}\PY{p}{,} \PY{n}{xy}\PY{o}{=}\PY{p}{(}\PY{n}{bar}\PY{o}{.}\PY{n}{get\PYZus{}width}\PY{p}{(}\PY{p}{)}\PY{p}{,} \PY{n}{bar}\PY{o}{.}\PY{n}{get\PYZus{}y}\PY{p}{(}\PY{p}{)}\PY{o}{+}\PY{n}{bar}\PY{o}{.}\PY{n}{get\PYZus{}height}\PY{p}{(}\PY{p}{)}\PY{o}{/}\PY{l+m+mi}{2}\PY{p}{)}\PY{p}{,}
                       \PY{n}{xytext}\PY{o}{=}\PY{p}{(}\PY{o}{\PYZhy{}}\PY{l+m+mi}{45}\PY{p}{,} \PY{l+m+mi}{0}\PY{p}{)}\PY{p}{,} \PY{n}{textcoords}\PY{o}{=}\PY{l+s+s1}{\PYZsq{}}\PY{l+s+s1}{offset pixels}\PY{l+s+s1}{\PYZsq{}}\PY{p}{,} \PY{n}{ha}\PY{o}{=}\PY{l+s+s2}{\PYZdq{}}\PY{l+s+s2}{left}\PY{l+s+s2}{\PYZdq{}}\PY{p}{,} \PY{n}{va}\PY{o}{=}\PY{l+s+s2}{\PYZdq{}}\PY{l+s+s2}{center}\PY{l+s+s2}{\PYZdq{}}\PY{p}{,} \PY{n}{color}\PY{o}{=}\PY{l+s+s1}{\PYZsq{}}\PY{l+s+s1}{white}\PY{l+s+s1}{\PYZsq{}}\PY{p}{)}
        
\PY{c+c1}{\PYZsh{} array for wanted fuel types }
\PY{n}{fuelTypes\PYZus{}label} \PY{o}{=} \PY{p}{[}\PY{l+s+s1}{\PYZsq{}}\PY{l+s+s1}{Diesel}\PY{l+s+s1}{\PYZsq{}}\PY{p}{,} \PY{l+s+s1}{\PYZsq{}}\PY{l+s+s1}{Super 95}\PY{l+s+s1}{\PYZsq{}}\PY{p}{,} \PY{l+s+s1}{\PYZsq{}}\PY{l+s+s1}{Super E10}\PY{l+s+s1}{\PYZsq{}}\PY{p}{]}

\PY{c+c1}{\PYZsh{} form changes and plotting}
\PY{k}{for} \PY{n}{i}\PY{p}{,} \PY{n}{fuelType\PYZus{}label} \PY{o+ow}{in} \PY{n+nb}{enumerate}\PY{p}{(}\PY{n}{fuelTypes\PYZus{}label}\PY{p}{)}\PY{p}{:}
    \PY{n}{ax}\PY{p}{[}\PY{n}{i}\PY{p}{]}\PY{o}{.}\PY{n}{set}\PY{p}{(}\PY{n}{title}\PY{o}{=}\PY{n}{fuelType\PYZus{}label}\PY{p}{,} \PY{n}{ylabel}\PY{o}{=} \PY{l+s+s1}{\PYZsq{}}\PY{l+s+s1}{Bundesländer}\PY{l+s+s1}{\PYZsq{}}\PY{p}{,} \PY{n}{xlabel}\PY{o}{=} \PY{l+s+s1}{\PYZsq{}}\PY{l+s+s1}{Preis in Euro}\PY{l+s+s1}{\PYZsq{}}\PY{p}{)}

\PY{n}{plt}\PY{o}{.}\PY{n}{tight\PYZus{}layout}\PY{p}{(}\PY{p}{)}    
\PY{n}{plt}\PY{o}{.}\PY{n}{show}\PY{p}{(}\PY{p}{)}
\end{Verbatim}
\end{tcolorbox}

    \begin{center}
\hspace{-2cm}
    \adjustimage{max size={0.9\linewidth}{0.9\paperheight}}{output_29_1.png}
    \end{center}

In den oberen Plots sind die östlichen Bundesländer durch den jeweils intensiveren Farbton hervorgehoben. Anhand des Plots lässt sich erkennen, dass die Treibstoffart Diesel in den östlichen Bundesländern vergleichsweise teurer ist. Startend mit Brandenburg und endend mit Sachsen-Anhalt belegen hierbei vier der fünf östlichen Bundesländer, mit der Ausnahme von Sachsen, die Plätze 3 bis 6 im Preisvergleich. Auf die Treibstoffgruppe Benzin, bestehend aus Super 95 und Super E10 lässt sich diese Erkenntnis jedoch nicht ausweiten. Hier gliedern sich die östlichen Bundesländer tendenziell in die günstigere Hälfte ein. 

\newpage
    \hypertarget{preisentwicklung-der-kraftstoffe}{%
\subsection{Preisentwicklung der
Kraftstoffe}\label{preisentwicklung-der-kraftstoffe}}

    Im Folgenden wollen wir die Preisentwicklung der Treibstoffarten
(Diesel, Super 95 und Super E10) aufzeigen. Dabei gehen wir gesondert
auf die Enwicklung der Preise unter Einfluss des Ukrainekiegs ein, so
dass wir den Zeitabschnitt ab 2021 vergrößert als Nebenvisualisierung
darstellen.

    \begin{tcolorbox}[breakable, size=fbox, boxrule=1pt, pad at break*=1mm,colback=cellbackground, colframe=cellborder]
\prompt{In}{incolor}{7}{\boxspacing}
\begin{Verbatim}[commandchars=\\\{\}]
\PY{c+c1}{\PYZsh{} creating df with wanted vars}
\PY{n}{preisentwicklung} \PY{o}{=} \PY{n}{timeseries}\PY{p}{[}\PY{p}{[}\PY{l+s+s1}{\PYZsq{}}\PY{l+s+s1}{date}\PY{l+s+s1}{\PYZsq{}}\PY{p}{,} \PY{l+s+s1}{\PYZsq{}}\PY{l+s+s1}{diesel}\PY{l+s+s1}{\PYZsq{}}\PY{p}{,} \PY{l+s+s1}{\PYZsq{}}\PY{l+s+s1}{e10}\PY{l+s+s1}{\PYZsq{}}\PY{p}{,} \PY{l+s+s1}{\PYZsq{}}\PY{l+s+s1}{e5}\PY{l+s+s1}{\PYZsq{}}\PY{p}{]}\PY{p}{]}

\PY{c+c1}{\PYZsh{} creating figure}
\PY{n}{fig}\PY{p}{,} \PY{n}{ax} \PY{o}{=} \PY{n}{plt}\PY{o}{.}\PY{n}{subplots}\PY{p}{(}\PY{l+m+mi}{1}\PY{p}{,}\PY{l+m+mi}{2}\PY{p}{,} \PY{n}{figsize}\PY{o}{=}\PY{p}{(}\PY{n}{width}\PY{p}{,}\PY{l+m+mi}{7}\PY{p}{)}\PY{p}{)}
\PY{n}{sns}\PY{o}{.}\PY{n}{set\PYZus{}style}\PY{p}{(}\PY{l+s+s1}{\PYZsq{}}\PY{l+s+s1}{whitegrid}\PY{l+s+s1}{\PYZsq{}}\PY{p}{)}

\PY{c+c1}{\PYZsh{} style an form changes }
\PY{n}{title} \PY{o}{=} \PY{n}{fig}\PY{o}{.}\PY{n}{suptitle}\PY{p}{(}\PY{l+s+s1}{\PYZsq{}}\PY{l+s+s1}{Preisentwicklung der Treibstoffarten}\PY{l+s+s1}{\PYZsq{}}\PY{p}{,} \PY{n}{fontsize}\PY{o}{=} \PY{n}{title\PYZus{}fontsize}\PY{p}{,} \PY{n}{fontweight}\PY{o}{=} \PY{l+s+s1}{\PYZsq{}}\PY{l+s+s1}{bold}\PY{l+s+s1}{\PYZsq{}}\PY{p}{)}
\PY{n}{title}\PY{o}{.}\PY{n}{set\PYZus{}x}\PY{p}{(}\PY{l+m+mf}{0.18}\PY{p}{)}

\PY{c+c1}{\PYZsh{} creating line plots }
\PY{k}{for} \PY{n}{fuelType}\PY{p}{,} \PY{n}{label}\PY{p}{,} \PY{n}{color} \PY{o+ow}{in} \PY{n+nb}{zip}\PY{p}{(}\PY{n}{fuelTypes}\PY{p}{,}\PY{n}{fuelTypes\PYZus{}label}\PY{p}{,} \PY{n}{colors}\PY{p}{)}\PY{p}{:}
    \PY{n}{sns}\PY{o}{.}\PY{n}{lineplot}\PY{p}{(}\PY{n}{data}\PY{o}{=}\PY{n}{preisentwicklung}\PY{p}{,} \PY{n}{x}\PY{o}{=}\PY{l+s+s1}{\PYZsq{}}\PY{l+s+s1}{date}\PY{l+s+s1}{\PYZsq{}}\PY{p}{,} \PY{n}{y}\PY{o}{=}\PY{n}{fuelType}\PY{p}{,} \PY{n}{label}\PY{o}{=}\PY{n}{label}\PY{p}{,}\PY{n}{ax}\PY{o}{=} \PY{n}{ax}\PY{p}{[}\PY{l+m+mi}{0}\PY{p}{]}\PY{p}{,} \PY{n}{color}\PY{o}{=} \PY{n}{color} \PY{p}{)}
    \PY{n}{sns}\PY{o}{.}\PY{n}{lineplot}\PY{p}{(}\PY{n}{data}\PY{o}{=}\PY{n}{preisentwicklung}\PY{p}{[}\PY{n}{preisentwicklung}\PY{o}{.}\PY{n}{date} \PY{o}{\PYZgt{}}\PY{o}{=} \PY{l+s+s1}{\PYZsq{}}\PY{l+s+s1}{2021\PYZhy{}01\PYZhy{}01}\PY{l+s+s1}{\PYZsq{}}\PY{p}{]}\PY{p}{,} 
                 \PY{n}{x}\PY{o}{=} \PY{l+s+s1}{\PYZsq{}}\PY{l+s+s1}{date}\PY{l+s+s1}{\PYZsq{}}\PY{p}{,} \PY{n}{y}\PY{o}{=} \PY{n}{fuelType}\PY{p}{,} \PY{n}{ax}\PY{o}{=} \PY{n}{ax}\PY{p}{[}\PY{l+m+mi}{1}\PY{p}{]}\PY{p}{,} \PY{n}{color}\PY{o}{=} \PY{n}{color} \PY{p}{)}
    
\PY{n}{line\PYZus{}label} \PY{o}{=} \PY{l+s+s1}{\PYZsq{}}\PY{l+s+s1}{Invasion Russlands}\PY{l+s+s1}{\PYZsq{}}    
\PY{n}{line\PYZus{}date} \PY{o}{=} \PY{n}{datetime}\PY{o}{.}\PY{n}{strptime}\PY{p}{(}\PY{l+s+s1}{\PYZsq{}}\PY{l+s+s1}{2022 02}\PY{l+s+s1}{\PYZsq{}}\PY{p}{,} \PY{l+s+s1}{\PYZsq{}}\PY{l+s+s1}{\PYZpc{}}\PY{l+s+s1}{Y }\PY{l+s+s1}{\PYZpc{}}\PY{l+s+s1}{m}\PY{l+s+s1}{\PYZsq{}}\PY{p}{)}
\PY{n}{ax}\PY{p}{[}\PY{l+m+mi}{0}\PY{p}{]}\PY{o}{.}\PY{n}{axvline}\PY{p}{(}\PY{n}{x}\PY{o}{=}\PY{n}{line\PYZus{}date}\PY{p}{,} \PY{n}{color}\PY{o}{=}\PY{l+s+s1}{\PYZsq{}}\PY{l+s+s1}{red}\PY{l+s+s1}{\PYZsq{}}\PY{p}{,} \PY{n}{linestyle}\PY{o}{=}\PY{l+s+s1}{\PYZsq{}}\PY{l+s+s1}{dotted}\PY{l+s+s1}{\PYZsq{}}\PY{p}{,} \PY{n}{label}\PY{o}{=} \PY{n}{line\PYZus{}label}\PY{p}{)}
\PY{n}{ax}\PY{p}{[}\PY{l+m+mi}{1}\PY{p}{]}\PY{o}{.}\PY{n}{axvline}\PY{p}{(}\PY{n}{x}\PY{o}{=}\PY{n}{line\PYZus{}date}\PY{p}{,} \PY{n}{color}\PY{o}{=}\PY{l+s+s1}{\PYZsq{}}\PY{l+s+s1}{red}\PY{l+s+s1}{\PYZsq{}}\PY{p}{,} \PY{n}{linestyle}\PY{o}{=}\PY{l+s+s1}{\PYZsq{}}\PY{l+s+s1}{dotted}\PY{l+s+s1}{\PYZsq{}}\PY{p}{)}

\PY{c+c1}{\PYZsh{} Add legend and style and form changes after plotting}
\PY{n}{ax}\PY{p}{[}\PY{l+m+mi}{0}\PY{p}{]}\PY{o}{.}\PY{n}{legend}\PY{p}{(}\PY{p}{)}
\PY{n}{ax}\PY{p}{[}\PY{l+m+mi}{0}\PY{p}{]}\PY{o}{.}\PY{n}{set}\PY{p}{(}\PY{n}{title}\PY{o}{=} \PY{l+s+s1}{\PYZsq{}}\PY{l+s+s1}{2014 \PYZhy{} Heute}\PY{l+s+s1}{\PYZsq{}}\PY{p}{,} \PY{n}{xlabel}\PY{o}{=}\PY{l+s+s1}{\PYZsq{}}\PY{l+s+s1}{Zeit}\PY{l+s+s1}{\PYZsq{}}\PY{p}{,} \PY{n}{ylabel}\PY{o}{=}\PY{l+s+s1}{\PYZsq{}}\PY{l+s+s1}{Preis pro Liter in Euro}\PY{l+s+s1}{\PYZsq{}}\PY{p}{)}
\PY{n}{ax}\PY{p}{[}\PY{l+m+mi}{1}\PY{p}{]}\PY{o}{.}\PY{n}{set}\PY{p}{(}\PY{n}{title}\PY{o}{=} \PY{l+s+s1}{\PYZsq{}}\PY{l+s+s1}{2021 \PYZhy{} Heute (Zoom Ukrainekrieg)}\PY{l+s+s1}{\PYZsq{}}\PY{p}{,} \PY{n}{xlabel}\PY{o}{=}\PY{l+s+s1}{\PYZsq{}}\PY{l+s+s1}{Zeit}\PY{l+s+s1}{\PYZsq{}}\PY{p}{,} \PY{n}{ylabel}\PY{o}{=}\PY{l+s+s1}{\PYZsq{}}\PY{l+s+s1}{\PYZsq{}}\PY{p}{)}

\PY{c+c1}{\PYZsh{} plotting}
\PY{n}{plt}\PY{o}{.}\PY{n}{tight\PYZus{}layout}\PY{p}{(}\PY{p}{)}
\PY{n}{plt}\PY{o}{.}\PY{n}{show}\PY{p}{(}\PY{p}{)}
\end{Verbatim}
\end{tcolorbox}

    \begin{center}
\hspace{-2cm}
    \adjustimage{max size={0.9\linewidth}{0.9\paperheight}}{output_33_0.png}
    \end{center}
    
    Die zwei oberen Grafiken veranschaulicht die Spritpreisentwicklungen der
3 Primärkraftstoffe Diesel, Super 95 und Super E10. Die x-Achse
beschreibt die Zeit in Jahren, während die y-Achse Aufschluss auf den
Preis pro Liter in Euro gibt. Der Grafik kann entnommen werden, dass bis
auf ein paar Ausnahmen Diesel der günstigere und Super 95 der teuerste
Kraftstoff ist.

Die granularere Darstellung der Achsen im rechten Plot gleicht einem
Zoom-Effekt und stellt somit den Zeitraum der Jahre 2021 bis heute dar.
Dieser Zeitraum ist wegen des starken Preisanstieges, ausgelöst durch
den Angriffskrieg Russlands auf die Ukraine und der Vielzahl an
Preiseinflüssen hervorzuheben. Anfang des Jahres 2022 ist ein Anstieg
der Spritpreise basierend auf dem Anstieg des Ölpreises auf Basis guter
Konjunkturaussichten, zu erkennen. Der auf den 24. Februar~2022 folgende
starke Spritpreisanstieg lässt sich wie folgt erklären. Die Angst vor
Lieferengpässen im Energiebereich auf Grund der damals drohenden
Sanktionen führte zur Preisexplosionen bei Öl und Gas, schließlich waren
damals die Ausmaße nicht bekannt. Die Angst vor Energieknappheit und die
damit verbundene erhöhte Zahlungsbereitschaft der Abnehmer führte bei
Investoren zu Spekulationen auf steigende Preise, welche natürlich
selbst in gewissen Maßen zur Preissteigerung beitrugen.


\newpage
   \hypertarget{anhang}{%
\section{Anhang}\label{anhang}}

\hypertarget{preprocessing}{%
\subsection{Preprocessing}
\label{preprocessing}}

Der gegebene Code durchläuft alle CSV-Dateien in einem bestimmten Ordner und konsolidiert ihre Daten in einem DataFrame. Die Daten werden dann aggregiert, um den durchschnittlichen (mean) Preis pro Station, Jahr und Monat zu berechnen. Der aggregierte Datensatz wird dann in einer neuen CSV-Datei gespeichert.

Der Import der Python-Bibliothek \textbf{glob} ermöglicht das Durchsuchen von Verzeichnissen nach Dateinamen oder -mustern. Mit der glob-Bibliothek können Dateilisten basierend auf bestimmten Kriterien z. B. alle Dateien mit einer bestimmten Erweiterung (* .txt) oder Dateien, die einem bestimmten Muster entsprechen (präfix_ *.csv),  erstellt werden. Sie bietet eine einfache Möglichkeit, Dateien in einem Verzeichnis zu filtern und auszuwählen, basierend auf ihren Namen oder anderen Eigenschaften. Die glob-Bibliothek ist besonders nützlich, wenn eine große Anzahl von Dateien vorliegt und nur eine bestimmte Auswahl davon verarbeiten möchten.


    \begin{tcolorbox}[breakable, size=fbox, boxrule=1pt, pad at break*=1mm,colback=cellbackground, colframe=cellborder]
\prompt{In}{incolor}{1}{\boxspacing}
\begin{Verbatim}[commandchars=\\\{\}]
\PY{k+kn}{import} \PY{n+nn}{glob}

\PY{c+c1}{\PYZsh{} Create an empty list to hold the data from each file}
\PY{n}{data\PYZus{}list} \PY{o}{=} \PY{p}{[}\PY{p}{]}

\PY{c+c1}{\PYZsh{} Loop over each file in the data folder}
\PY{k}{for} \PY{n}{filename} \PY{o+ow}{in} \PY{n}{glob}\PY{o}{.}\PY{n}{glob}\PY{p}{(}\PY{l+s+s1}{\PYZsq{}}\PY{l+s+s1}{*.csv}\PY{l+s+s1}{\PYZsq{}}\PY{p}{)}\PY{p}{:}
    
    \PY{c+c1}{\PYZsh{} Read the CSV file into a DataFrame}
    \PY{n}{data} \PY{o}{=} \PY{n}{pd}\PY{o}{.}\PY{n}{read\PYZus{}csv}\PY{p}{(}\PY{n}{filename}\PY{p}{,} \PY{n}{parse\PYZus{}dates}\PY{o}{=}\PY{p}{[}\PY{l+s+s1}{\PYZsq{}}\PY{l+s+s1}{date}\PY{l+s+s1}{\PYZsq{}}\PY{p}{]}\PY{p}{)}
    
    \PY{c+c1}{\PYZsh{} Append the data to the l}
    \PY{c+c1}{\PYZsh{} ist}
    \PY{n}{data\PYZus{}list}\PY{o}{.}\PY{n}{append}\PY{p}{(}\PY{n}{data}\PY{p}{)}
    
\PY{c+c1}{\PYZsh{} Concatenate the data in the list into a single DataFrame}
\PY{n}{all\PYZus{}data} \PY{o}{=} \PY{n}{pd}\PY{o}{.}\PY{n}{concat}\PY{p}{(}\PY{n}{data\PYZus{}list}\PY{p}{)}

\PY{c+c1}{\PYZsh{} Convert the date column to a datetime object and extract the month}
\PY{n}{all\PYZus{}data}\PY{p}{[}\PY{l+s+s1}{\PYZsq{}}\PY{l+s+s1}{date}\PY{l+s+s1}{\PYZsq{}}\PY{p}{]} \PY{o}{=} \PY{n}{pd}\PY{o}{.}\PY{n}{to\PYZus{}datetime}\PY{p}{(}\PY{n}{all\PYZus{}data}\PY{p}{[}\PY{l+s+s1}{\PYZsq{}}\PY{l+s+s1}{date}\PY{l+s+s1}{\PYZsq{}}\PY{p}{]}\PY{p}{,} \PY{n}{utc}\PY{o}{=}\PY{k+kc}{True}\PY{p}{)}
\PY{n}{all\PYZus{}data}\PY{p}{[}\PY{l+s+s1}{\PYZsq{}}\PY{l+s+s1}{month}\PY{l+s+s1}{\PYZsq{}}\PY{p}{]} \PY{o}{=} \PY{n}{all\PYZus{}data}\PY{p}{[}\PY{l+s+s1}{\PYZsq{}}\PY{l+s+s1}{date}\PY{l+s+s1}{\PYZsq{}}\PY{p}{]}\PY{o}{.}\PY{n}{dt}\PY{o}{.}\PY{n}{strftime}\PY{p}{(}\PY{l+s+s1}{\PYZsq{}}\PY{l+s+s1}{\PYZpc{}}\PY{l+s+s1}{m}\PY{l+s+s1}{\PYZsq{}}\PY{p}{)}
\PY{n}{all\PYZus{}data}\PY{p}{[}\PY{l+s+s1}{\PYZsq{}}\PY{l+s+s1}{year}\PY{l+s+s1}{\PYZsq{}}\PY{p}{]} \PY{o}{=} \PY{n}{all\PYZus{}data}\PY{p}{[}\PY{l+s+s1}{\PYZsq{}}\PY{l+s+s1}{date}\PY{l+s+s1}{\PYZsq{}}\PY{p}{]}\PY{o}{.}\PY{n}{dt}\PY{o}{.}\PY{n}{strftime}\PY{p}{(}\PY{l+s+s1}{\PYZsq{}}\PY{l+s+s1}{\PYZpc{}}\PY{l+s+s1}{Y}\PY{l+s+s1}{\PYZsq{}}\PY{p}{)}

\PY{c+c1}{\PYZsh{} Group by station\PYZus{}id, year, and month then take the mean price}
\PY{n}{agg\PYZus{}data} \PY{o}{=} \PY{n}{all\PYZus{}data}\PY{o}{.}\PY{n}{groupby}\PY{p}{(}\PY{p}{[}\PY{l+s+s1}{\PYZsq{}}\PY{l+s+s1}{station\PYZus{}uuid}\PY{l+s+s1}{\PYZsq{}}\PY{p}{,} \PY{l+s+s1}{\PYZsq{}}\PY{l+s+s1}{year}\PY{l+s+s1}{\PYZsq{}}\PY{p}{,}\PY{l+s+s1}{\PYZsq{}}\PY{l+s+s1}{month}\PY{l+s+s1}{\PYZsq{}}\PY{p}{]}\PY{p}{)}\PY{o}{.}\PY{n}{agg}\PY{p}{(}\PY{p}{\PYZob{}}\PY{l+s+s1}{\PYZsq{}}\PY{l+s+s1}{diesel}\PY{l+s+s1}{\PYZsq{}}\PY{p}{:} \PY{l+s+s1}{\PYZsq{}}\PY{l+s+s1}{mean}\PY{l+s+s1}{\PYZsq{}}\PY{p}{,} \PY{l+s+s1}{\PYZsq{}}\PY{l+s+s1}{e5}\PY{l+s+s1}{\PYZsq{}}\PY{p}{:} \PY{l+s+s1}{\PYZsq{}}\PY{l+s+s1}{mean}\PY{l+s+s1}{\PYZsq{}}\PY{p}{,} \PY{l+s+s1}{\PYZsq{}}\PY{l+s+s1}{e10}\PY{l+s+s1}{\PYZsq{}}\PY{p}{:} \PY{l+s+s1}{\PYZsq{}}\PY{l+s+s1}{mean}\PY{l+s+s1}{\PYZsq{}}\PY{p}{\PYZcb{}}\PY{p}{)}\PY{o}{.}\PY{n}{round}\PY{p}{(}\PY{l+m+mi}{3}\PY{p}{)}\PY{o}{.}\PY{n}{reset\PYZus{}index}\PY{p}{(}\PY{p}{)}

\PY{c+c1}{\PYZsh{} Save the aggregated data to a new CSV file}
\PY{n}{agg\PYZus{}data}\PY{o}{.}\PY{n}{to\PYZus{}csv}\PY{p}{(}\PY{l+s+s1}{\PYZsq{}}\PY{l+s+s1}{./aggregated\PYZus{}data.csv}\PY{l+s+s1}{\PYZsq{}}\PY{p}{,} \PY{n}{index}\PY{o}{=}\PY{k+kc}{False}\PY{p}{)}

\PY{c+c1}{\PYZsh{} Create an empty list to hold the data from each file}
\PY{n}{data\PYZus{}list} \PY{o}{=} \PY{p}{[}\PY{p}{]}

\PY{c+c1}{\PYZsh{} Loop over each file in the data folder}
\PY{k}{for} \PY{n}{filename} \PY{o+ow}{in} \PY{n}{glob}\PY{o}{.}\PY{n}{glob}\PY{p}{(}\PY{l+s+sa}{r}\PY{l+s+s1}{\PYZsq{}}\PY{l+s+s1}{C:}\PY{l+s+s1}{\PYZbs{}}\PY{l+s+s1}{Users}\PY{l+s+s1}{\PYZbs{}}\PY{l+s+s1}{ahmed}\PY{l+s+s1}{\PYZbs{}}\PY{l+s+s1}{Desktop}\PY{l+s+s1}{\PYZbs{}}\PY{l+s+s1}{Data Science}\PY{l+s+s1}{\PYZbs{}}\PY{l+s+s1}{Git Repo}\PY{l+s+s1}{\PYZbs{}}\PY{l+s+s1}{datavisualisation}\PY{l+s+s1}{\PYZbs{}}\PY{l+s+s1}{aggregated}\PY{l+s+s1}{\PYZbs{}}\PY{l+s+s1}{*.csv}\PY{l+s+s1}{\PYZsq{}}\PY{p}{)}\PY{p}{:}
    
    \PY{c+c1}{\PYZsh{} Read the CSV file into a DataFrame}
    \PY{n}{data} \PY{o}{=} \PY{n}{pd}\PY{o}{.}\PY{n}{read\PYZus{}csv}\PY{p}{(}\PY{n}{filename}\PY{p}{)}
    
    \PY{c+c1}{\PYZsh{} Append the data to the list}
    \PY{n}{data\PYZus{}list}\PY{o}{.}\PY{n}{append}\PY{p}{(}\PY{n}{data}\PY{p}{)}

\PY{c+c1}{\PYZsh{} Concatenate the data in the list into a single DataFrame}
\PY{n}{combined\PYZus{}data} \PY{o}{=} \PY{n}{pd}\PY{o}{.}\PY{n}{concat}\PY{p}{(}\PY{n}{data\PYZus{}list}\PY{p}{)}

\PY{c+c1}{\PYZsh{} Save the combined data to a new CSV file}
\PY{n}{combined\PYZus{}data}\PY{o}{.}\PY{n}{to\PYZus{}csv}\PY{p}{(}\PY{l+s+s1}{\PYZsq{}}\PY{l+s+s1}{./agg\PYZus{}timeseries.csv}\PY{l+s+s1}{\PYZsq{}}\PY{p}{,} \PY{n}{index}\PY{o}{=}\PY{k+kc}{False}\PY{p}{)}
\end{Verbatim}
\end{tcolorbox}

    % Add a bibliography block to the postdoc
    
    
    
\end{document}
